\documentclass{beamer}
\usetheme{Madrid}
\usecolortheme{dolphin}
\usepackage{amsmath}
\usepackage{graphicx}
\usepackage{tikz}
\usepackage{booktabs}

\title{Black-Scholes Mathematical Analysis}
\subtitle{Partial Differential Equations and Option Pricing}
\author{Lecturer}
\date{\today}

\begin{document}

\begin{frame}
\titlepage
\end{frame}

\begin{frame}
\frametitle{Outline}
\tableofcontents
\end{frame}

\section{Proving the Black-Scholes Equation is Parabolic}

\begin{frame}
\frametitle{The Black-Scholes Partial Differential Equation}
The Black-Scholes equation for option pricing is:

$$\frac{\partial V}{\partial t} + \frac{1}{2}\sigma^2 S^2 \frac{\partial^2 V}{\partial S^2} + rS \frac{\partial V}{\partial S} - rV = 0$$

Where:
\begin{itemize}
\item $V(S,t)$ = option value
\item $S$ = underlying asset price
\item $t$ = time
\item $\sigma$ = volatility
\item $r$ = risk-free interest rate
\end{itemize}
\end{frame}

\begin{frame}
\frametitle{General Form of Second-Order PDEs}
A general second-order PDE in two variables has the form:

$$A\frac{\partial^2 u}{\partial x^2} + B\frac{\partial^2 u}{\partial x \partial y} + C\frac{\partial^2 u}{\partial y^2} + D\frac{\partial u}{\partial x} + E\frac{\partial u}{\partial y} + F \cdot u + G = 0$$

\begin{block}{Classification of PDEs}
PDEs are classified based on the discriminant $\Delta = B^2 - 4AC$:
\begin{itemize}
\item \textbf{Elliptic}: $\Delta < 0$
\item \textbf{Parabolic}: $\Delta = 0$
\item \textbf{Hyperbolic}: $\Delta > 0$
\end{itemize}
\end{block}
\end{frame}

\begin{frame}
\frametitle{Identifying Coefficients in Black-Scholes}
Rewriting the Black-Scholes equation in standard form with $x = S$ and $y = t$:

$$\frac{1}{2}\sigma^2 S^2 \frac{\partial^2 V}{\partial S^2} + 0 \frac{\partial^2 V}{\partial S \partial t} + 0 \frac{\partial^2 V}{\partial t^2} + rS \frac{\partial V}{\partial S} + \frac{\partial V}{\partial t} - rV = 0$$

\begin{block}{Coefficients}
\begin{itemize}
\item $A = \frac{1}{2}\sigma^2 S^2$
\item $B = 0$
\item $C = 0$
\item $D = rS$
\item $E = 1$
\item $F = -r$
\end{itemize}
\end{block}
\end{frame}

\begin{frame}
\frametitle{Computing the Discriminant}
$$\Delta = B^2 - 4AC = 0^2 - 4 \cdot \left(\frac{1}{2}\sigma^2 S^2\right) \cdot 0 = 0 - 0 = 0$$

\begin{block}{Conclusion}
Since the discriminant $\Delta = 0$, the Black-Scholes equation is \textbf{parabolic}.
\end{block}
\end{frame}

\begin{frame}
\frametitle{Physical Interpretation}
The parabolic nature reflects:

\begin{enumerate}
\item \textbf{Diffusion Process}: The $\sigma^2 S^2 \frac{\partial^2 V}{\partial S^2}$ term represents diffusion in the asset price, similar to heat diffusion

\item \textbf{Time Evolution}: Information propagates through the system over time, characteristic of parabolic PDEs

\item \textbf{No Second Time Derivative}: Unlike wave equations (hyperbolic), there's no "acceleration" term in time
\end{enumerate}

\begin{block}{Important Implications}
\begin{itemize}
\item Parabolic PDEs have unique solutions under appropriate boundary conditions
\item Numerical methods for parabolic PDEs are well-established
\item The solution exhibits smoothing properties typical of diffusion processes
\end{itemize}
\end{block}
\end{frame}

\end{document}