\documentclass[aspectratio=169]{beamer}
\usetheme{metropolis}
\usepackage{amsmath}
\usepackage{graphicx}
\usepackage{tikz}
\usepackage{booktabs}

\title{Black-Scholes Mathematical Analysis}
\subtitle{Partial Differential Equations and Option Pricing}
\author{Dr. Juliho Castillo}
\date{\today}

\begin{document}

\begin{frame}
\titlepage
\end{frame}

\begin{frame}
\frametitle{Outline}
\tableofcontents
\end{frame}

\section{Proving the Black-Scholes Equation is Parabolic}

\begin{frame}
\frametitle{The Black-Scholes PDE}
The Black-Scholes equation for option pricing is:

\[\frac{\partial V}{\partial t} + \frac{1}{2}\sigma^2 S^2 \frac{\partial^2 V}{\partial S^2} + rS \frac{\partial V}{\partial S} - rV = 0\]
\end{frame}

\begin{frame}
\frametitle{Variables in the Black-Scholes Equation}
Where:
\begin{itemize}
\item \(V(S,t)\) = option value
\item \(S\) = underlying asset price
\item \(t\) = time
\item \(\sigma\) = volatility
\item \(r\) = risk-free interest rate
\end{itemize}
\end{frame}

\begin{frame}
\frametitle{General Form of Second-Order PDEs}
A general second-order PDE in two variables has the form:

\[A\frac{\partial^2 u}{\partial x^2} + B\frac{\partial^2 u}{\partial x \partial y} + C\frac{\partial^2 u}{\partial y^2} + D\frac{\partial u}{\partial x} + E\frac{\partial u}{\partial y} + F \cdot u + G = 0\]
\end{frame}

\begin{frame}
\frametitle{Classification of PDEs}
\begin{block}{PDE Classification}
PDEs are classified based on the discriminant \(\Delta = B^2 - 4AC\):
\begin{itemize}
\item \textbf{Elliptic}: \(\Delta < 0\)
\item \textbf{Parabolic}: \(\Delta = 0\)
\item \textbf{Hyperbolic}: \(\Delta > 0\)
\end{itemize}
\end{block}
\end{frame}

\begin{frame}
\frametitle{Black-Scholes in Standard Form}
Rewriting the Black-Scholes equation in standard form with \(x = S\) and \(y = t\):

\[\frac{1}{2}\sigma^2 S^2 \frac{\partial^2 V}{\partial S^2} + 0 \frac{\partial^2 V}{\partial S \partial t} + 0 \frac{\partial^2 V}{\partial t^2} + rS \frac{\partial V}{\partial S} + \frac{\partial V}{\partial t} - rV = 0\]
\end{frame}

\begin{frame}
\frametitle{Identifying the Coefficients}
\begin{block}{Coefficients}
\begin{itemize}
\item \(A = \frac{1}{2}\sigma^2 S^2\)
\item \(B = 0\)
\item \(C = 0\)
\item \(D = rS\)
\item \(E = 1\)
\item \(F = -r\)
\end{itemize}
\end{block}
\end{frame}

\begin{frame}
\frametitle{Computing the Discriminant}
\[\Delta = B^2 - 4AC = 0^2 - 4 \cdot \left(\frac{1}{2}\sigma^2 S^2\right) \cdot 0 = 0\]

\begin{block}{Conclusion}
Since the discriminant \(\Delta = 0\), the Black-Scholes equation is \textbf{parabolic}.
\end{block}
\end{frame}

\begin{frame}
\frametitle{Physical Interpretation - Diffusion Process}
The parabolic nature reflects a \textbf{diffusion process}:

The \(\sigma^2 S^2 \frac{\partial^2 V}{\partial S^2}\) term represents diffusion in the asset price, similar to heat diffusion.
\end{frame}

\begin{frame}
\frametitle{Physical Interpretation - Time Evolution}
\textbf{Time Evolution}: Information propagates through the system over time, characteristic of parabolic PDEs.

\textbf{No Second Time Derivative}: Unlike wave equations (hyperbolic), there's no "acceleration" term in time.
\end{frame}

\begin{frame}
\frametitle{Important Implications}
\begin{block}{Mathematical Properties}
\begin{itemize}
\item Parabolic PDEs have unique solutions under appropriate boundary conditions
\item Numerical methods for parabolic PDEs are well-established
\item The solution exhibits smoothing properties typical of diffusion processes
\end{itemize}
\end{block}
\end{frame}

\section{Itô's Formula and Stochastic Calculus}

\begin{frame}
\frametitle{Why Do We Need Stochastic Calculus?}
When dealing with random processes like stock prices, ordinary calculus fails because:
\begin{itemize}
\item Brownian motion isn't differentiable
\item The quadratic term \((dx)^2\) doesn't vanish
\end{itemize}
\end{frame}

\begin{frame}
\frametitle{We Need Special Rules}
We need special rules for stochastic differentiation to handle random processes properly.
\end{frame}

\begin{frame}
\frametitle{Brownian Motion - Starting Point}
For Brownian motion \(W(t)\):
\begin{itemize}
\item \(W(0) = 0\) (starts at zero)
\item Independent, normally distributed increments
\end{itemize}
\end{frame}

\begin{frame}
\frametitle{Brownian Motion - Key Properties}
\begin{itemize}
\item Continuous paths but nowhere differentiable
\item \textbf{Key property}: \((dW)^2 = dt\)
\end{itemize}

This last property is fundamental to stochastic calculus!
\end{frame}

\begin{frame}
\frametitle{Ordinary vs. Stochastic Chain Rule}
\textbf{Ordinary calculus}: 
\[df = \frac{\partial f}{\partial t}dt + \frac{\partial f}{\partial x}dx\]

\textbf{Stochastic calculus}: We get an extra term!
\end{frame}

\begin{frame}
\frametitle{Itô's Formula: The Stochastic Chain Rule}
For a function \(f(W,t)\) where \(W\) is Brownian motion:

\[df = \left(\frac{\partial f}{\partial t} + \frac{1}{2} \frac{\partial^2 f}{\partial W^2}\right) dt + \frac{\partial f}{\partial W} dW\]
\end{frame}

\begin{frame}
\frametitle{The Key Insight}
\begin{block}{Why the Extra Term?}
The second derivative term appears because \((dW)^2 = dt \neq 0\) in stochastic calculus!
\end{block}

In ordinary calculus, \((dx)^2 = 0\), but in stochastic calculus, \((dW)^2 = dt\).
\end{frame}

\begin{frame}
\frametitle{Stock Price Model}
Stock prices follow geometric Brownian motion (GBM):
\[dS = \mu S dt + \sigma S dW\]
\end{frame}

\begin{frame}
\frametitle{GBM Parameters}
Where:
\begin{itemize}
\item \(\mu\) = expected return (drift)
\item \(\sigma\) = volatility
\item \(dW\) = Brownian motion increment
\end{itemize}
\end{frame}

\begin{frame}
\frametitle{Why Geometric Brownian Motion?}
\begin{itemize}
\item Stock prices can't go negative
\item Percentage changes are more natural than absolute changes
\item Leads to lognormal distribution of future prices
\end{itemize}
\end{frame}

\begin{frame}
\frametitle{Itô's Formula for Option Values}
For a function \(V(S,t)\) where \(S\) follows GBM, Itô's formula gives:

\[dV = \left(\frac{\partial V}{\partial t} + \mu S \frac{\partial V}{\partial S} + \frac{1}{2}\sigma^2 S^2 \frac{\partial^2 V}{\partial S^2}\right) dt + \sigma S \frac{\partial V}{\partial S} dW\]
\end{frame}

\begin{frame}
\frametitle{The Derivation Logic}
\begin{enumerate}
\item Start with Taylor expansion around \((S,t)\)
\item Use \(dS = \mu S dt + \sigma S dW\)
\item Compute \((dS)^2 = \sigma^2 S^2 dt\) (key step!)
\item Keep terms up to order \(dt\)
\end{enumerate}
\end{frame}

\begin{frame}
\frametitle{Fundamental Stochastic Rules}
\begin{block}{The Three Key Rules}
\begin{itemize}
\item \((dt)^2 = 0\) - second-order infinitesimals vanish
\item \((dW)^2 = dt\) - \textbf{this is the key stochastic rule}
\item \(dt \times dW = 0\) - deterministic and random parts are orthogonal
\end{itemize}
\end{block}
\end{frame}

\begin{frame}
\frametitle{Computing \((dS)^2\) - Step 1}
\[(dS)^2 = (\mu S dt + \sigma S dW)^2\]

Expanding:
\[(dS)^2 = \mu^2 S^2 (dt)^2 + 2\mu\sigma S^2 (dt)(dW) + \sigma^2 S^2 (dW)^2\]
\end{frame}

\begin{frame}
\frametitle{Computing \((dS)^2\) - Step 2}
Applying the stochastic rules:
\begin{align}
(dS)^2 &= \mu^2 S^2 (dt)^2 + 2\mu\sigma S^2 (dt)(dW) + \sigma^2 S^2 (dW)^2 \\
&= 0 + 0 + \sigma^2 S^2 dt \\
&= \sigma^2 S^2 dt
\end{align}
\end{frame}

\begin{frame}
\frametitle{Why Itô's Formula Matters - Same Randomness}
The formula tells us how option value \(V(S,t)\) changes:

\[dV = \left[\frac{\partial V}{\partial t} + \mu S \frac{\partial V}{\partial S} + \frac{1}{2}\sigma^2 S^2 \frac{\partial^2 V}{\partial S^2}\right] dt + \sigma S \frac{\partial V}{\partial S} dW\]

\textbf{Key observation}: Both stock and option driven by the same \(dW\)!
\end{frame}

\begin{frame}
\frametitle{Why Itô's Formula Matters - Components}
\begin{itemize}
\item \textbf{The drift term}: Contains the volatility effect \(\frac{1}{2}\sigma^2 S^2 \frac{\partial^2 V}{\partial S^2}\)
\item \textbf{The diffusion term}: Proportional to \(\frac{\partial V}{\partial S}\) (the delta!)
\end{itemize}
\end{frame}

\begin{frame}
\frametitle{Why Itô's Formula Matters - Hedging}
\textbf{Foundation for hedging}: This shared randomness enables delta hedging.

The same \(dW\) term in both stock and option changes allows us to create hedged portfolios.
\end{frame}

\section{Derivation of the Black-Scholes Equation}

\begin{frame}
\frametitle{Key Assumptions - Market Model}
\begin{enumerate}
\item \textbf{Geometric Brownian Motion}: Stock price follows
\[dS = \mu S dt + \sigma S dW\]
where \(dW\) is a Wiener process (Brownian motion)

\item \textbf{Constant Parameters}: Risk-free rate \(r\), volatility \(\sigma\) are constant
\end{enumerate}
\end{frame}

\begin{frame}
\frametitle{Key Assumptions - Market Conditions}
\begin{enumerate}
\setcounter{enumi}{2}
\item \textbf{No Dividends}: The stock pays no dividends

\item \textbf{Perfect Market}: No transaction costs, continuous trading, unlimited borrowing/lending at rate \(r\)

\item \textbf{European Exercise}: Option can only be exercised at expiration
\end{enumerate}
\end{frame}

\begin{frame}
\frametitle{Step 1: Stock Price Model}
The stock price follows geometric Brownian motion:
\[dS = \mu S dt + \sigma S dW\]

Where \(\mu\) is the expected return (drift) and \(\sigma\) is the volatility.
\end{frame}

\begin{frame}
\frametitle{Step 1: Interpretation}
\begin{block}{Economic Meaning}
The stock has a deterministic trend \(\mu S\) and random fluctuations \(\sigma S\).
\end{block}
\end{frame}

\begin{frame}
\frametitle{Step 2: Apply Itô's Lemma}
For option value \(V(S,t)\), applying Itô's lemma:

\[dV = \left(\frac{\partial V}{\partial t} + \mu S \frac{\partial V}{\partial S} + \frac{1}{2}\sigma^2 S^2 \frac{\partial^2 V}{\partial S^2}\right) dt + \sigma S \frac{\partial V}{\partial S} dW\]
\end{frame}

\begin{frame}
\frametitle{Step 2: Key Insight}
\begin{block}{Shared Randomness}
Both \(dS\) and \(dV\) contain the same random term \(dW\) - this is crucial for hedging.
\end{block}
\end{frame}

\begin{frame}
\frametitle{Step 3: The Hedged Portfolio}
Create a portfolio \(\Pi\) consisting of:
\begin{itemize}
\item \textbf{Long 1 option} (value = \(V\))
\item \textbf{Short \(\Delta\) shares} (value = \(-\Delta S\))
\end{itemize}

So: \(\Pi = V - \Delta S\)
\end{frame}

\begin{frame}
\frametitle{Step 3: Portfolio Change}
The change in portfolio value is:
\[d\Pi = dV - \Delta \, dS\]
\end{frame}

\begin{frame}
\frametitle{Step 4: Substituting the Expressions}
Substitute the expressions for \(dV\) and \(dS\):

\begin{align}
d\Pi &= \left[\frac{\partial V}{\partial t} + \mu S \frac{\partial V}{\partial S} + \frac{1}{2}\sigma^2 S^2 \frac{\partial^2 V}{\partial S^2} - \Delta\mu S\right] dt \\
&\quad + \left[\sigma S \frac{\partial V}{\partial S} - \Delta\sigma S\right] dW
\end{align}
\end{frame}

\begin{frame}
\frametitle{Step 4: Factoring the Random Term}
Factoring the \(dW\) coefficient:
\[\text{dW coefficient} = \sigma S\left[\frac{\partial V}{\partial S} - \Delta\right]\]
\end{frame}

\begin{frame}
\frametitle{Step 5: The Magic Choice}
To eliminate the random \(dW\) term, we set:
\[\Delta = \frac{\partial V}{\partial S}\]

This is the \textbf{delta hedge ratio}!
\end{frame}

\begin{frame}
\frametitle{Step 5: Why This Works}
\begin{block}{Perfect Hedge Logic}
\(\frac{\partial V}{\partial S}\) is the option's sensitivity to stock price changes. By shorting exactly this many shares:
\begin{itemize}
\item Stock goes up \(\rightarrow\) option gains, short stock loses
\item Stock goes down \(\rightarrow\) option loses, short stock gains
\item \textbf{Net effect: zero risk!}
\end{itemize}
\end{block}
\end{frame}

\begin{frame}
\frametitle{Step 5: Risk-Free Portfolio}
With this choice: 
\[d\Pi = \left[\frac{\partial V}{\partial t} + \frac{1}{2}\sigma^2 S^2 \frac{\partial^2 V}{\partial S^2}\right] dt\]

The portfolio change is now deterministic (no \(dW\) term)!
\end{frame}

\begin{frame}
\frametitle{Step 6: No-Arbitrage Condition}
Since the portfolio is now risk-free, it must earn the risk-free rate \(r\):
\[d\Pi = r\Pi \, dt\]
\end{frame}

\begin{frame}
\frametitle{Step 6: Portfolio Value}
But \(\Pi = V - \Delta S = V - \frac{\partial V}{\partial S}S\)

Therefore:
\[\left[\frac{\partial V}{\partial t} + \frac{1}{2}\sigma^2 S^2 \frac{\partial^2 V}{\partial S^2}\right] dt = r\left[V - \frac{\partial V}{\partial S}S\right] dt\]
\end{frame}

\begin{frame}
\frametitle{Step 7: Dividing by \(dt\)}
Dividing by \(dt\) and rearranging:

\[\frac{\partial V}{\partial t} + \frac{1}{2}\sigma^2 S^2 \frac{\partial^2 V}{\partial S^2} = rV - rS\frac{\partial V}{\partial S}\]
\end{frame}

\begin{frame}
\frametitle{Step 7: The Black-Scholes Equation}
Rearranging to standard form:

\[\frac{\partial V}{\partial t} + \frac{1}{2}\sigma^2 S^2 \frac{\partial^2 V}{\partial S^2} + rS\frac{\partial V}{\partial S} - rV = 0\]

\textbf{We have derived the Black-Scholes equation!}
\end{frame}

\begin{frame}
\frametitle{Remarkable Result 1}
\textbf{The expected return \(\mu\) disappeared completely!}

Option prices depend only on volatility \(\sigma\), not expected return.
\end{frame}

\begin{frame}
\frametitle{Remarkable Result 2}
\textbf{Perfect hedging creates risk-free portfolio.}

The mathematics shows that continuous rebalancing eliminates all risk.
\end{frame}

\begin{frame}
\frametitle{The Deep Intuition - Matching Sensitivities}
\begin{block}{What We Accomplished}
By setting \(\Delta = \frac{\partial V}{\partial S}\), we matched the sensitivities:
\begin{itemize}
\item Option position sensitivity = Stock position sensitivity
\item Random fluctuations cancel out perfectly
\item Only deterministic terms remain
\end{itemize}
\end{block}
\end{frame}

\begin{frame}
\frametitle{The Deep Intuition - Economic Principle}
\begin{block}{Economic Interpretation}
In an efficient market with no arbitrage opportunities:
\begin{itemize}
\item Risk-free portfolios must earn the risk-free rate
\item This constraint determines option prices uniquely
\item The mathematics captures this economic principle
\end{itemize}
\end{block}
\end{frame}

\end{document}