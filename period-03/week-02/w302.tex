\documentclass{beamer}
\usetheme{Madrid}
\usecolortheme{dolphin}
\usepackage{amsmath}
\usepackage{graphicx}
\usepackage{tikz}
\usepackage{booktabs}

\title{Black-Scholes Mathematical Analysis}
\subtitle{Partial Differential Equations and Option Pricing}
\author{Lecturer}
\date{\today}

\begin{document}

\begin{frame}
\titlepage
\end{frame}

\begin{frame}
\frametitle{Outline}
\tableofcontents
\end{frame}

\section{Proving the Black-Scholes Equation is Parabolic}

\begin{frame}
\frametitle{The Black-Scholes Partial Differential Equation}
The Black-Scholes equation for option pricing is:

$$\frac{\partial V}{\partial t} + \frac{1}{2}\sigma^2 S^2 \frac{\partial^2 V}{\partial S^2} + rS \frac{\partial V}{\partial S} - rV = 0$$

Where:
\begin{itemize}
\item $V(S,t)$ = option value
\item $S$ = underlying asset price
\item $t$ = time
\item $\sigma$ = volatility
\item $r$ = risk-free interest rate
\end{itemize}
\end{frame}

\begin{frame}
\frametitle{General Form of Second-Order PDEs}
A general second-order PDE in two variables has the form:

$$A\frac{\partial^2 u}{\partial x^2} + B\frac{\partial^2 u}{\partial x \partial y} + C\frac{\partial^2 u}{\partial y^2} + D\frac{\partial u}{\partial x} + E\frac{\partial u}{\partial y} + F \cdot u + G = 0$$

\begin{block}{Classification of PDEs}
PDEs are classified based on the discriminant $\Delta = B^2 - 4AC$:
\begin{itemize}
\item \textbf{Elliptic}: $\Delta < 0$
\item \textbf{Parabolic}: $\Delta = 0$
\item \textbf{Hyperbolic}: $\Delta > 0$
\end{itemize}
\end{block}
\end{frame}

\begin{frame}
\frametitle{Identifying Coefficients in Black-Scholes}
Rewriting the Black-Scholes equation in standard form with $x = S$ and $y = t$:

$$\frac{1}{2}\sigma^2 S^2 \frac{\partial^2 V}{\partial S^2} + 0 \frac{\partial^2 V}{\partial S \partial t} + 0 \frac{\partial^2 V}{\partial t^2} + rS \frac{\partial V}{\partial S} + \frac{\partial V}{\partial t} - rV = 0$$

\begin{block}{Coefficients}
\begin{itemize}
\item $A = \frac{1}{2}\sigma^2 S^2$
\item $B = 0$
\item $C = 0$
\item $D = rS$
\item $E = 1$
\item $F = -r$
\end{itemize}
\end{block}
\end{frame}

\begin{frame}
\frametitle{Computing the Discriminant}
$$\Delta = B^2 - 4AC = 0^2 - 4 \cdot \left(\frac{1}{2}\sigma^2 S^2\right) \cdot 0 = 0 - 0 = 0$$

\begin{block}{Conclusion}
Since the discriminant $\Delta = 0$, the Black-Scholes equation is \textbf{parabolic}.
\end{block}
\end{frame}

\begin{frame}
\frametitle{Physical Interpretation}
The parabolic nature reflects:

\begin{enumerate}
\item \textbf{Diffusion Process}: The $\sigma^2 S^2 \frac{\partial^2 V}{\partial S^2}$ term represents diffusion in the asset price, similar to heat diffusion

\item \textbf{Time Evolution}: Information propagates through the system over time, characteristic of parabolic PDEs

\item \textbf{No Second Time Derivative}: Unlike wave equations (hyperbolic), there's no "acceleration" term in time
\end{enumerate}

\begin{block}{Important Implications}
\begin{itemize}
\item Parabolic PDEs have unique solutions under appropriate boundary conditions
\item Numerical methods for parabolic PDEs are well-established
\item The solution exhibits smoothing properties typical of diffusion processes
\end{itemize}
\end{block}
\end{frame}

\section{Itô's Formula and Stochastic Calculus}

\begin{frame}
\frametitle{The Need for Stochastic Calculus}
\begin{block}{Problem with Ordinary Calculus}
When dealing with random processes like stock prices, ordinary calculus fails because:
\begin{itemize}
\item Brownian motion isn't differentiable
\item The quadratic term $(dx)^2$ doesn't vanish
\item We need special rules for stochastic differentiation
\end{itemize}
\end{block}

\begin{block}{Brownian Motion Properties}
For Brownian motion $W(t)$:
\begin{itemize}
\item $W(0) = 0$ (starts at zero)
\item Independent, normally distributed increments
\item Continuous paths but nowhere differentiable
\item \textbf{Key property}: $(dW)^2 = dt$
\end{itemize}
\end{block}
\end{frame}

\begin{frame}
\frametitle{Itô's Formula: The Stochastic Chain Rule}
For a function $f(W,t)$ where $W$ is Brownian motion:

$df = \left(\frac{\partial f}{\partial t} + \frac{1}{2} \frac{\partial^2 f}{\partial W^2}\right) dt + \frac{\partial f}{\partial W} dW$

\begin{block}{Key Insight}
The second derivative term appears because $(dW)^2 = dt \neq 0$ in stochastic calculus!
\end{block}

In ordinary calculus: $df = \frac{\partial f}{\partial t}dt + \frac{\partial f}{\partial x}dx$

In stochastic calculus: We get an extra $\frac{1}{2}\frac{\partial^2 f}{\partial W^2}dt$ term.
\end{frame}

\begin{frame}
\frametitle{Geometric Brownian Motion}
Stock prices follow geometric Brownian motion (GBM):
$dS = \mu S dt + \sigma S dW$

Where:
\begin{itemize}
\item $\mu$ = expected return (drift)
\item $\sigma$ = volatility
\item $dW$ = Brownian motion increment
\end{itemize}

\begin{block}{Why GBM?}
\begin{itemize}
\item Stock prices can't go negative
\item Percentage changes are more natural than absolute changes
\item Leads to lognormal distribution of future prices
\end{itemize}
\end{block}
\end{frame}

\begin{frame}
\frametitle{Itô's Formula for Functions of GBM}
For a function $V(S,t)$ where $S$ follows GBM, Itô's formula gives:

$dV = \left(\frac{\partial V}{\partial t} + \mu S \frac{\partial V}{\partial S} + \frac{1}{2}\sigma^2 S^2 \frac{\partial^2 V}{\partial S^2}\right) dt + \sigma S \frac{\partial V}{\partial S} dW$

\begin{block}{The Derivation Logic}
\begin{enumerate}
\item Start with Taylor expansion around $(S,t)$
\item Use $dS = \mu S dt + \sigma S dW$
\item Compute $(dS)^2 = \sigma^2 S^2 dt$ (key step!)
\item Keep terms up to order $dt$
\end{enumerate}
\end{block}
\end{frame}

\begin{frame}
\frametitle{The Crucial Stochastic Rules}
\begin{block}{Fundamental Rules}
\begin{itemize}
\item $(dt)^2 = 0$ - second-order infinitesimals vanish
\item $(dW)^2 = dt$ - \textbf{this is the key stochastic rule}
\item $dt \times dW = 0$ - deterministic and random parts are orthogonal
\end{itemize}
\end{block}

\begin{block}{Computing $(dS)^2$}
$(dS)^2 = (\mu S dt + \sigma S dW)^2$
$= \mu^2 S^2 (dt)^2 + 2\mu\sigma S^2 (dt)(dW) + \sigma^2 S^2 (dW)^2$
$= 0 + 0 + \sigma^2 S^2 dt = \sigma^2 S^2 dt$
\end{block}
\end{frame}

\begin{frame}
\frametitle{Why Itô's Formula Matters for Options}
The formula tells us how option value $V(S,t)$ changes:

$dV = \left[\frac{\partial V}{\partial t} + \mu S \frac{\partial V}{\partial S} + \frac{1}{2}\sigma^2 S^2 \frac{\partial^2 V}{\partial S^2}\right] dt + \sigma S \frac{\partial V}{\partial S} dW$

\begin{block}{Key Observations}
\begin{itemize}
\item \textbf{Same randomness}: Both stock and option driven by the same $dW$
\item \textbf{The drift term}: Contains the volatility effect $\frac{1}{2}\sigma^2 S^2 \frac{\partial^2 V}{\partial S^2}$
\item \textbf{The diffusion term}: Proportional to $\frac{\partial V}{\partial S}$ (the delta!)
\item \textbf{Foundation for hedging}: This shared randomness enables delta hedging
\end{itemize}
\end{block}
\end{frame}

\section{Derivation of the Black-Scholes Equation}

\begin{frame}
\frametitle{Key Assumptions for Black-Scholes}
\begin{enumerate}
\item \textbf{Geometric Brownian Motion}: Stock price follows
$$dS = \mu S dt + \sigma S dW$$
where $dW$ is a Wiener process (Brownian motion)

\item \textbf{Constant Parameters}: Risk-free rate $r$, volatility $\sigma$ are constant

\item \textbf{No Dividends}: The stock pays no dividends

\item \textbf{Perfect Market}: No transaction costs, continuous trading, unlimited borrowing/lending at rate $r$

\item \textbf{European Exercise}: Option can only be exercised at expiration
\end{enumerate}
\end{frame}

\begin{frame}
\frametitle{Step 1: Stock Price Model}
The stock price follows geometric Brownian motion:
$$dS = \mu S dt + \sigma S dW$$

Where:
\begin{itemize}
\item $\mu$ is the expected return (drift)
\item $\sigma$ is the volatility  
\item $dW$ is a Wiener process (random walk)
\end{itemize}

\begin{block}{Interpretation}
The stock has a deterministic trend $\mu S$ and random fluctuations $\sigma S$.
\end{block}
\end{frame}

\begin{frame}
\frametitle{Step 2: Option Value Changes (Itô's Lemma)}
For option value $V(S,t)$, applying Itô's lemma:

$$dV = \left(\frac{\partial V}{\partial t} + \mu S \frac{\partial V}{\partial S} + \frac{1}{2}\sigma^2 S^2 \frac{\partial^2 V}{\partial S^2}\right) dt + \sigma S \frac{\partial V}{\partial S} dW$$

\begin{block}{Key Insight}
Both $dS$ and $dV$ contain the same random term $dW$ - this is crucial for hedging.
\end{block}
\end{frame}

\begin{frame}
\frametitle{Step 3: Constructing the Hedged Portfolio}
Create a portfolio $\Pi$ consisting of:
\begin{itemize}
\item \textbf{Long 1 option} (value = $V$)
\item \textbf{Short $\Delta$ shares} (value = $-\Delta S$)
\end{itemize}

So: $\Pi = V - \Delta S$

The change in portfolio value is:
$$d\Pi = dV - \Delta \, dS$$
\end{frame}

\begin{frame}
\frametitle{Step 4: The Critical Substitution}
Substitute the expressions for $dV$ and $dS$:

\begin{align}
d\Pi &= \left[\frac{\partial V}{\partial t} + \mu S \frac{\partial V}{\partial S} + \frac{1}{2}\sigma^2 S^2 \frac{\partial^2 V}{\partial S^2} - \Delta\mu S\right] dt \\
&\quad + \left[\sigma S \frac{\partial V}{\partial S} - \Delta\sigma S\right] dW
\end{align}

Factoring the $dW$ coefficient:
$$\text{dW coefficient} = \sigma S\left[\frac{\partial V}{\partial S} - \Delta\right]$$
\end{frame}

\begin{frame}
\frametitle{Step 4: Eliminating Risk - The Magic Choice}
To eliminate the random $dW$ term, we set:
$$\Delta = \frac{\partial V}{\partial S}$$

\begin{block}{Why This Works}
$\frac{\partial V}{\partial S}$ is the option's sensitivity to stock price changes. By shorting exactly this many shares, we create perfect hedge:
\begin{itemize}
\item Stock goes up $\rightarrow$ option gains, short stock loses
\item Stock goes down $\rightarrow$ option loses, short stock gains
\item \textbf{Net effect: zero risk!}
\end{itemize}
\end{block}

With this choice: $d\Pi = \left[\frac{\partial V}{\partial t} + \frac{1}{2}\sigma^2 S^2 \frac{\partial^2 V}{\partial S^2}\right] dt$
\end{frame}

\begin{frame}
\frametitle{Step 5: No-Arbitrage Condition}
Since the portfolio is now risk-free, it must earn the risk-free rate $r$:
$$d\Pi = r\Pi \, dt$$

But $\Pi = V - \Delta S = V - \frac{\partial V}{\partial S}S$

Therefore:
$$\left[\frac{\partial V}{\partial t} + \frac{1}{2}\sigma^2 S^2 \frac{\partial^2 V}{\partial S^2}\right] dt = r\left[V - \frac{\partial V}{\partial S}S\right] dt$$
\end{frame}

\begin{frame}
\frametitle{Step 6: The Black-Scholes Equation Emerges}
Dividing by $dt$ and rearranging:

$$\frac{\partial V}{\partial t} + \frac{1}{2}\sigma^2 S^2 \frac{\partial^2 V}{\partial S^2} = rV - rS\frac{\partial V}{\partial S}$$

Rearranging to standard form:

$$\frac{\partial V}{\partial t} + \frac{1}{2}\sigma^2 S^2 \frac{\partial^2 V}{\partial S^2} + rS\frac{\partial V}{\partial S} - rV = 0$$

\begin{block}{Remarkable Results}
\begin{itemize}
\item The expected return $\mu$ disappeared completely!
\item Option prices depend only on volatility $\sigma$, not expected return
\item Perfect hedging creates risk-free portfolio
\end{itemize}
\end{block}
\end{frame}

\begin{frame}
\frametitle{The Deep Intuition}
\begin{block}{What We Accomplished}
By setting $\Delta = \frac{\partial V}{\partial S}$, we matched the sensitivities:
\begin{itemize}
\item Option position sensitivity = Stock position sensitivity
\item Random fluctuations cancel out perfectly
\item Only deterministic terms remain
\end{itemize}
\end{block}

\begin{block}{Economic Interpretation}
In an efficient market with no arbitrage opportunities:
\begin{itemize}
\item Risk-free portfolios must earn the risk-free rate
\item This constraint determines option prices uniquely
\item The mathematics captures this economic principle
\end{itemize}
\end{block}
\end{frame}

\end{document}