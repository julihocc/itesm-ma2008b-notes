\documentclass[12pt,a4paper]{article}
\usepackage[utf8]{inputenc}
\usepackage[T1]{fontenc}
\usepackage[english]{babel}
\usepackage{amsmath}
\usepackage{amsfonts}
\usepackage{amssymb}
\usepackage[margin=2.5cm]{geometry}
\usepackage{booktabs}
\usepackage{graphicx}
\usepackage{float}
\usepackage{hyperref}
\usepackage{listings}
\usepackage{xcolor}
\usepackage{algorithm}
\usepackage{algorithmic}
\usepackage{lmodern}
\usepackage{microtype}
\usepackage{mathtools}
\usepackage{siunitx}
\usepackage[capitalise]{cleveref}
\usepackage{caption}
\captionsetup{font=small,labelfont=bf,skip=5pt}

% Custom notation macros
\newcommand{\V}{\mathbf{V}}
\newcommand{\Lop}{\mathcal{L}}
% Grid parameter macros
\newcommand{\DS}{\Delta S}
\newcommand{\Dt}{\Delta t}
\newcommand{\M}{M}
\newcommand{\N}{N}

% Code styling
\lstset{
    basicstyle=\ttfamily\footnotesize,
    commentstyle=\color{gray},
    keywordstyle=\color{blue},
    stringstyle=\color{red},
    breaklines=true,
    showstringspaces=false,
    frame=single,
    numbers=left,
    numberstyle=\tiny\color{gray},
    captionpos=b
}

% Algorithm styling for CLRS style
\renewcommand{\algorithmiccomment}[1]{\hfill $\triangleright$ #1}
\renewcommand{\algorithmicrequire}{\textbf{Input:}}
\renewcommand{\algorithmicensure}{\textbf{Output:}}

% Configure algorithm numbering to follow subsection structure
\renewcommand{\thealgorithm}{\thesubsection.\arabic{algorithm}}
\numberwithin{algorithm}{subsection}

\title{\textbf{Partial Differential Equations in Finance: \\
A Comprehensive Study of Black-Scholes Numerical Methods} \\
\large Period 03 - Complete Analysis and Results}
\author{Dr. Juliho David Castillo Colmenares}
\date{Updated: \today}

\begin{document}

\maketitle

\begin{abstract}
This comprehensive article presents a complete analysis of partial differential equations in finance, focusing on the Black-Scholes equation and its numerical solution methods. The study covers theoretical foundations, mathematical analysis, and practical implementation of finite difference methods for option pricing. Through detailed comparison of explicit, implicit, and Crank–Nicolson finite difference schemes, we demonstrate that all numerical methods achieve excellent accuracy (> \SI{99.5}{\percent}) when properly implemented with refined grids. The article synthesizes results from five weeks of intensive study, providing both theoretical insights and practical computational results.
\end{abstract}

\tableofcontents
\listoffigures
\newpage
\listoftables
\newpage

\section{Introduction}
\label{sec:introduction}

The application of partial differential equations (PDEs) in financial mathematics has revolutionized modern option pricing theory. The Black-Scholes equation, derived by Fischer Black, Myron Scholes, and Robert Merton in the early 1970s, provides a mathematical framework for pricing European options and serves as the foundation for modern derivatives pricing.

This comprehensive study examines the Black-Scholes equation from multiple perspectives: its mathematical properties, analytical solutions, and numerical approximation methods. We present a thorough analysis of finite difference methods, comparing their accuracy, stability, and computational efficiency.

\subsection{Scope and Objectives}

This article addresses the following key objectives:

\begin{itemize}
\item \textbf{Mathematical Foundation}: Establish the Black-Scholes PDE and prove its parabolic nature
\item \textbf{Analytical Solutions}: Derive and implement the exact Black-Scholes formula
\item \textbf{Numerical Methods}: Implement and analyze three finite difference schemes
\item \textbf{Comparative Analysis}: Evaluate accuracy and performance of all methods
\item \textbf{Practical Applications}: Provide computational results and best practices
\end{itemize}

\section*{Notation}
\begin{tabular}{ll}
  $S$ & Underlying asset price \\
  $K$ & Strike price \\
  $T$ & Time to maturity \\
  $t_j$ & Time grid point, $t_j = j\,\Delta t$ \\
  $S_i$ & Price grid point, $S_i = i\,\Delta S$ \\
  $\Delta S$ & Price step size, $\Delta S = S_{\max}/M$ \\
  $\Delta t$ & Time step size, $\Delta t = T/N$ \\
  $M$ & Number of spatial intervals \\
  $N$ & Number of time intervals \\
  $\sigma$ & Volatility of the underlying asset \\
  $r$ & Risk-free interest rate \\
\end{tabular}

\section{Mathematical Foundation}
\label{sec:math-foundation}

\subsection{The Black-Scholes Partial Differential Equation}

The Black-Scholes equation for European option pricing is given by:

\begin{equation}
\frac{\partial V}{\partial t} + \frac{1}{2}\sigma^2 S^2 \frac{\partial^2 V}{\partial S^2} + rS \frac{\partial V}{\partial S} - rV = 0
\end{equation}

where:
\begin{itemize}
\item $V(S,t)$ is the option value as a function of stock price $S$ and time $t$
\item $\sigma$ is the volatility of the underlying asset
\item $r$ is the risk-free interest rate
\item $T$ is the expiration time
\end{itemize}

\subsection{Classification as a Parabolic PDE}

To classify the Black-Scholes equation, we rewrite it in the standard form:

\begin{equation}
A\frac{\partial^2 V}{\partial S^2} + B\frac{\partial^2 V}{\partial S \partial t} + C\frac{\partial^2 V}{\partial t^2} + D\frac{\partial V}{\partial S} + E\frac{\partial V}{\partial t} + FV = 0
\end{equation}

Comparing with the Black-Scholes equation:
\begin{align}
A &= \frac{1}{2}\sigma^2 S^2 \\
B &= 0 \\
C &= 0 \\
D &= rS \\
E &= 1 \\
F &= -r
\end{align}

The discriminant is:
\begin{equation}
\Delta = B^2 - 4AC = 0^2 - 4 \cdot \frac{1}{2}\sigma^2 S^2 \cdot 0 = 0
\end{equation}

Since $\Delta = 0$, the Black-Scholes equation is \textbf{parabolic}, which has important implications for solution methods and boundary conditions.

\subsection{Boundary and Initial Conditions}

For a European call option with strike price $K$ and expiration time $T$:

\textbf{Terminal condition (at $t = T$):}
\begin{equation}
V(S, T) = \max(S - K, 0)
\end{equation}

\textbf{Boundary conditions:}
\begin{align}
V(0, t) &= 0 \quad \text{(call worthless when $S = 0$)} \\
V(S, t) &\sim S - Ke^{-r(T-t)} \quad \text{as } S \to \infty
\end{align}

\section{Analytical Solution}
\label{sec:analytical-solution}

\subsection{The Black-Scholes Formula}

The analytical solution for a European call option is:

\begin{equation}
V(S_0, 0) = S_0 N(d_1) - Ke^{-rT} N(d_2)
\end{equation}

where:
\begin{align}
d_1 &= \frac{\ln(S_0/K) + (r + \sigma^2/2)T}{\sigma\sqrt{T}} \\
d_2 &= d_1 - \sigma\sqrt{T}
\end{align}

and $N(\cdot)$ is the cumulative standard normal distribution function.

\subsection{Reference Example Calculation}

Using the standard test parameters:
\begin{itemize}
\item $S_0 = \SI{100}{}$ (current stock price)
\item $K = \SI{100}{}$ (strike price)
\item $T = \num{0.25}$ years (3 months to expiration)
\item $r = \SI{5}{\percent}$ (risk-free rate)
\item $\sigma = \SI{20}{\percent}$ volatility
\end{itemize}

\textbf{Step-by-step calculation (convert numbers):}
\begin{align}
d_1 &= \frac{\ln(\num{100}/\num{100}) + (\num{0.05} + \num{0.20}^2/2) \times \num{0.25}}{\num{0.20} \times \sqrt{\num{0.25}}} \\
    &= \frac{0 + (\num{0.05} + \num{0.02}) \times \num{0.25}}{\num{0.20} \times \num{0.5}} \\
    &= \frac{\num{0.0175}}{\num{0.10}} = \num{0.175}
\end{align}

\begin{equation}
d_2 = \num{0.175} - \num{0.20} \times \num{0.5} = \num{0.075}
\end{equation}

\begin{align}
N(d_1) &= N(\num{0.175}) = \num{0.5695} \\
N(d_2) &= N(\num{0.075}) = \num{0.5299}
\end{align}

\begin{align}
V_{BS} &= \num{100} \times \num{0.5695} - \num{100} \times e^{-\num{0.05} \times \num{0.25}} \times \num{0.5299} \\
       &= \num{56.95} - \num{100} \times \num{0.9876} \times \num{0.5299} \\
       &= \num{56.95} - \num{52.33} = \SI{4.6150}{\dollar}
\end{align}

This analytical result serves as our benchmark for evaluating numerical methods.

\section{Finite Difference Methods}
\label{sec:fdm-methods}

Finite difference methods approximate the continuous Black-Scholes PDE using discrete grid points in space and time. We implement three primary schemes: explicit, implicit, and Crank-Nicolson.

\subsection{Grid Setup}

All numerical methods use the following discretization:

\begin{itemize}
\item \textbf{Stock price domain}: $[0, S_{max}]$ with $S_{max} = \num{200}$
\item \textbf{Spatial grid}: $M = \num{100}$ intervals, giving $\Delta S = S_{max}/M = \num{2.0}$
\item \textbf{Time domain}: $[0, T]$ with $T = 0.25$
\item \textbf{Temporal grid}: $N = \num{1000}$ intervals, giving $\Delta t = T/N = \num{0.00025}$
\end{itemize}

Grid points are defined as:
\begin{align}
S_i &= i \cdot \Delta S, \quad i = 0, 1, \ldots, M \\
t_j &= j \cdot \Delta t, \quad j = 0, 1, \ldots, N
\end{align}

\subsection{Explicit Finite Difference Method}

\subsubsection{Mathematical Formulation}

The explicit method uses forward differences in time and central differences in space:

\begin{equation}
\frac{V_i^{j+1} - V_i^j}{\Delta t} + \frac{1}{2}\sigma^2 S_i^2 \frac{V_{i+1}^j - 2V_i^j + V_{i-1}^j}{(\Delta S)^2} + rS_i \frac{V_{i+1}^j - V_{i-1}^j}{2\Delta S} - rV_i^j = 0
\end{equation}

Solving for $V_i^{j+1}$:

\begin{equation}
V_i^{j+1} = V_i^j + \Delta t \left[ \frac{1}{2}\sigma^2 S_i^2 \frac{V_{i+1}^j - 2V_i^j + V_{i-1}^j}{(\Delta S)^2} + rS_i \frac{V_{i+1}^j - V_{i-1}^j}{2\Delta S} - rV_i^j \right]
\end{equation}

\subsubsection{Stability Analysis}

The explicit method requires the following stability condition:

\begin{equation}
\Delta t \leq \frac{(\Delta S)^2}{2\sigma^2 S_{max}^2 + r(\Delta S)^2}
\end{equation}

For our parameters, this gives $\Delta t \leq 0.000248$, which our choice of $\Delta t = 0.00025$ satisfies.

\subsubsection{Implementation Results}

\textbf{Computational Details:}
\begin{itemize}
\item Algorithm complexity: $O(MN)$ operations
\item Memory requirement: $O(M)$ for solution vector
\item Computation time: $\sim\SI{0.124}{\second}$
\end{itemize}

\textbf{Numerical Results:}
\begin{itemize}
\item Option value: $V(S_0, 0) = \$4.5955$
\item Absolute error: $|4.5955 - 4.6150| = \$0.0195$
\item Relative error: \SI{0.42}{\percent}
\end{itemize}

\subsubsection{Algorithm Description}

The following presents the explicit finite difference method in structured algorithmic form.

\begin{algorithm}[H]
\caption{Explicit Finite Difference for Black-Scholes}
\begin{algorithmic}[1]
\REQUIRE $S_0$ (initial stock price), $K$ (strike price), $T$ (time to expiration), $r$ (risk-free rate), $\sigma$ (volatility), $S_{max}$ (maximum stock price), $M$ (spatial grid points), $N$ (temporal grid points)
\ENSURE Option value $V(S_0, 0)$

\STATE $\Delta S \leftarrow S_{max} / M$
\STATE $\Delta t \leftarrow T / N$
\STATE \textsc{Validate-Stability}$(\Delta t, \Delta S, \sigma, S_{max}, r)$
\STATE $V \leftarrow$ \textsc{Initialize-Grid}$(M, K, S_{max})$
\STATE \textsc{Set-Boundary-Conditions}$(V, M, K, r, T, S_{max})$

\FOR{$j = N-1$ \TO $0$}
    \STATE $t \leftarrow j \cdot \Delta t$
    \FOR{$i = 1$ \TO $M-1$}
        \STATE $S_i \leftarrow i \cdot \Delta S$
        \STATE $V[i] \leftarrow$ \textsc{Explicit-Update}$(V, i, S_i, \Delta t, \Delta S, r, \sigma)$
    \ENDFOR
    \STATE \textsc{Update-Boundary-Conditions}$(V, M, K, r, t, S_{max})$
\ENDFOR

\RETURN \textsc{Interpolate-Value}$(V, S_0, \Delta S)$
\end{algorithmic}
\end{algorithm}

\begin{algorithm}[H]
\caption{Validate-Stability}
\begin{algorithmic}[1]
\REQUIRE $\Delta t$, $\Delta S$, $\sigma$, $S_{max}$, $r$
\ENSURE Stability condition verified

\STATE $\Delta t_{max} \leftarrow \frac{(\Delta S)^2}{2\sigma^2 S_{max}^2 + r(\Delta S)^2}$
\IF{$\Delta t > \Delta t_{max}$}
    \STATE \textbf{error} "Stability condition violated: $\Delta t$ too large"
\ENDIF
\STATE \textbf{print} "Stability condition satisfied: $\Delta t \leq \Delta t_{max}$"
\end{algorithmic}
\end{algorithm}

\begin{algorithm}[H]
\caption{Initialize-Grid}
\begin{algorithmic}[1]
\REQUIRE $M$ (number of spatial points), $K$ (strike price), $S_{max}$ (maximum stock price)
\ENSURE Array $V$ of size $M+1$ with terminal conditions

\STATE $V \leftarrow$ new array of size $(M+1)$
\STATE $\Delta S \leftarrow S_{max} / M$

\FOR{$i = 0$ \TO $M$}
    \STATE $S_i \leftarrow i \cdot \Delta S$
    \STATE $V[i] \leftarrow \max(S_i - K, 0)$ \COMMENT{Terminal payoff}
\ENDFOR

\RETURN $V$
\end{algorithmic}
\end{algorithm}

\begin{algorithm}[H]
\caption{Set-Boundary-Conditions}
\begin{algorithmic}[1]
\REQUIRE $V$ (option values array), $M$, $K$, $r$, $T$, $S_{max}$
\ENSURE Boundary conditions applied

\STATE $V[0] \leftarrow 0$ \COMMENT{Lower boundary: option worthless when $S = 0$}
\STATE $V[M] \leftarrow S_{max} - K \cdot e^{-rT}$ \COMMENT{Upper boundary: intrinsic value}
\end{algorithmic}
\end{algorithm}

\begin{algorithm}[H]
\caption{Explicit-Update}
\begin{algorithmic}[1]
\REQUIRE $V$ (current values), $i$ (spatial index), $S_i$ (stock price), $\Delta t$, $\Delta S$, $r$, $\sigma$
\ENSURE Updated option value $V_{new}$

\STATE $\alpha \leftarrow \frac{\Delta t}{2} \left( \frac{\sigma^2 S_i^2}{(\Delta S)^2} - \frac{rS_i}{\Delta S} \right)$
\STATE $\beta \leftarrow 1 - \Delta t \left( \frac{\sigma^2 S_i^2}{(\Delta S)^2} + r \right)$
\STATE $\gamma \leftarrow \frac{\Delta t}{2} \left( \frac{\sigma^2 S_i^2}{(\Delta S)^2} + \frac{rS_i}{\Delta S} \right)$

\STATE $V_{new} \leftarrow \alpha \cdot V[i-1] + \beta \cdot V[i] + \gamma \cdot V[i+1]$

\RETURN $V_{new}$
\end{algorithmic}
\end{algorithm}

\begin{algorithm}[H]
\caption{Update-Boundary-Conditions}
\begin{algorithmic}[1]
\REQUIRE $V$, $M$, $K$, $r$, $t$ (current time), $S_{max}$
\ENSURE Boundary conditions updated for current time step

\STATE $V[0] \leftarrow 0$ \COMMENT{Lower boundary remains zero}
\STATE $V[M] \leftarrow S_{max} - K \cdot e^{-r(T-t)}$ \COMMENT{Upper boundary with time decay}
\end{algorithmic}
\end{algorithm}

\begin{algorithm}[H]
\caption{Interpolate-Value}
\begin{algorithmic}[1]
\REQUIRE $V$ (final option values), $S_0$ (target stock price), $\Delta S$
\ENSURE Interpolated option value at $S_0$

\STATE $index \leftarrow S_0 / \Delta S$
\STATE $i \leftarrow \lfloor index \rfloor$
\STATE $fraction \leftarrow index - i$

\IF{$i \geq$ length$(V) - 1$}
    \RETURN $V[$length$(V) - 1]$ \COMMENT{Boundary case}
\ENDIF

\STATE $V_{interp} \leftarrow (1 - fraction) \cdot V[i] + fraction \cdot V[i+1]$
\RETURN $V_{interp}$
\end{algorithmic}
\end{algorithm}

\paragraph{Complexity Analysis}

\textbf{Time Complexity:} The main algorithm executes $N$ time steps, each processing $M-1$ spatial points with constant-time updates, yielding $O(N \cdot M)$ total complexity.

\textbf{Space Complexity:} The algorithm requires $O(M)$ space for the option values array, with all other variables using $O(1)$ space.

\subsection{Implicit Finite Difference Method}

\subsubsection{Mathematical Formulation}

The implicit method uses backward differences in time and central differences in space:

\begin{equation}
\frac{V_i^{j+1} - V_i^j}{\Delta t} + \frac{1}{2}\sigma^2 S_i^2 \frac{V_{i+1}^{j+1} - 2V_i^{j+1} + V_{i-1}^{j+1}}{(\Delta S)^2} + rS_i \frac{V_{i+1}^{j+1} - V_{i-1}^{j+1}}{2\Delta S} - rV_i^{j+1} = 0
\end{equation}

This leads to a tridiagonal system of equations at each time step:

\begin{equation}
\mathbf{A} \mathbf{V}^{j+1} = \mathbf{V}^j
\end{equation}

where $\mathbf{A}$ is a tridiagonal matrix with coefficients:
\begin{align}
a_i &= -\frac{\Delta t}{2} \left( \frac{\sigma^2 S_i^2}{(\Delta S)^2} - \frac{rS_i}{\Delta S} \right) \\
b_i &= 1 + \Delta t \left( \frac{\sigma^2 S_i^2}{(\Delta S)^2} + r \right) \\
c_i &= -\frac{\Delta t}{2} \left( \frac{\sigma^2 S_i^2}{(\Delta S)^2} + \frac{rS_i}{\Delta S} \right)
\end{align}

\subsubsection{Stability Analysis}

The implicit method is unconditionally stable, meaning there are no restrictions on the time step size for stability. This allows for larger time steps and faster computation.

\subsubsection{Implementation Results}

\textbf{Computational Details:}
\begin{itemize}
\item Algorithm complexity: $O(MN)$ operations (with efficient tridiagonal solver)
\item Memory requirement: $O(M)$ for solution vector and matrix storage
\item Computation time: $\sim\SI{0.052}{\second}$
\end{itemize}

\textbf{Numerical Results:}
\begin{itemize}
\item Option value: $V(S_0, 0) = \$4.5944$
\item Absolute error: $|4.5944 - 4.6150| = \$0.0206$
\item Relative error: $0.45\%$
\end{itemize}

\subsubsection{Algorithm Description}

The following presents the implicit finite difference method in structured algorithmic form.

\begin{algorithm}[H]
\caption{Implicit Finite Difference for Black-Scholes}
\begin{algorithmic}[1]
\REQUIRE $S_0$ (initial stock price), $K$ (strike price), $T$ (time to expiration), $r$ (risk-free rate), $\sigma$ (volatility), $S_{max}$ (maximum stock price), $M$ (spatial grid points), $N$ (temporal grid points)
\ENSURE Option value $V(S_0, 0)$

\STATE $\Delta S \leftarrow S_{max} / M$
\STATE $\Delta t \leftarrow T / N$
\STATE $V \leftarrow$ \textsc{Initialize-Grid}$(M, K, S_{max})$
\STATE $A \leftarrow$ \textsc{Build-Tridiagonal-Matrix}$(M, \Delta t, \Delta S, \sigma, r, S_{max})$
\STATE \textsc{Set-Boundary-Conditions}$(V, M, K, r, T, S_{max})$

\FOR{$j = N-1$ \TO $0$}
    \STATE $t \leftarrow j \cdot \Delta t$
    \STATE $V \leftarrow$ \textsc{Solve-Tridiagonal-System}$(A, V)$
    \STATE \textsc{Update-Boundary-Conditions}$(V, M, K, r, t, S_{max})$
\ENDFOR

\RETURN \textsc{Interpolate-Value}$(V, S_0, \Delta S)$
\end{algorithmic}
\end{algorithm}

\begin{algorithm}[H]
\caption{Build-Tridiagonal-Matrix}
\begin{algorithmic}[1]
\REQUIRE $M$ (spatial points), $\Delta t$, $\Delta S$, $\sigma$, $r$, $S_{max}$
\ENSURE Tridiagonal matrix $A$ of size $(M-1) \times (M-1)$

\STATE $A \leftarrow$ new tridiagonal matrix of size $(M-1) \times (M-1)$

\FOR{$i = 1$ \TO $M-1$}
    \STATE $S_i \leftarrow i \cdot \Delta S$
    \STATE $\alpha \leftarrow \frac{\Delta t}{2} \left( \frac{\sigma^2 S_i^2}{(\Delta S)^2} - \frac{rS_i}{\Delta S} \right)$
    \STATE $\beta \leftarrow 1 + \Delta t \left( \frac{\sigma^2 S_i^2}{(\Delta S)^2} + r \right)$
    \STATE $\gamma \leftarrow \frac{\Delta t}{2} \left( \frac{\sigma^2 S_i^2}{(\Delta S)^2} + \frac{rS_i}{\Delta S} \right)$
    
    \IF{$i > 1$}
        \STATE $A[i][i-1] \leftarrow -\alpha$ \COMMENT{Lower diagonal}
    \ENDIF
    \STATE $A[i][i] \leftarrow \beta$ \COMMENT{Main diagonal}
    \IF{$i < M-1$}
        \STATE $A[i][i+1] \leftarrow -\gamma$ \COMMENT{Upper diagonal}
    \ENDIF
\ENDFOR

\RETURN $A$
\end{algorithmic}
\end{algorithm}

\begin{algorithm}[H]
\caption{Solve-Tridiagonal-System}
\begin{algorithmic}[1]
\REQUIRE Tridiagonal matrix $A$, right-hand side vector $b$
\ENSURE Solution vector $x$ such that $Ax = b$

\STATE $n \leftarrow$ size$(b)$
\STATE $c' \leftarrow$ new array of size $n$
\STATE $d' \leftarrow$ new array of size $n$

\COMMENT{Forward elimination}
\STATE $c'[1] \leftarrow A[1][2] / A[1][1]$
\STATE $d'[1] \leftarrow b[1] / A[1][1]$

\FOR{$i = 2$ \TO $n-1$}
    \STATE $m \leftarrow A[i][i] - A[i][i-1] \cdot c'[i-1]$
    \STATE $c'[i] \leftarrow A[i][i+1] / m$
    \STATE $d'[i] \leftarrow (b[i] - A[i][i-1] \cdot d'[i-1]) / m$
\ENDFOR

\STATE $m \leftarrow A[n][n] - A[n][n-1] \cdot c'[n-1]$
\STATE $d'[n] \leftarrow (b[n] - A[n][n-1] \cdot d'[n-1]) / m$

\COMMENT{Back substitution}
\STATE $x[n] \leftarrow d'[n]$
\FOR{$i = n-1$ \TO $1$}
    \STATE $x[i] \leftarrow d'[i] - c'[i] \cdot x[i+1]$
\ENDFOR

\RETURN $x$
\end{algorithmic}
\end{algorithm}

\paragraph{Complexity Analysis}

\textbf{Time Complexity:} The main algorithm executes $N$ time steps, each requiring $O(M)$ operations for matrix assembly and $O(M)$ operations for tridiagonal system solution, yielding $O(N \cdot M)$ total complexity.

\textbf{Space Complexity:} Requires $O(M)$ storage for the solution vector and $O(M)$ for the tridiagonal matrix coefficients.

\subsection{Crank-Nicolson Method}

\subsubsection{Mathematical Formulation}

The Crank-Nicolson method combines explicit and implicit approaches using the average of spatial operators at consecutive time levels:

\begin{equation}
\frac{V_i^{j+1} - V_i^j}{\Delta t} + \frac{1}{2} \left[ L(V_i^{j+1}) + L(V_i^j) \right] = 0
\end{equation}

where $L$ is the spatial differential operator. This leads to the matrix equation:

\begin{equation}
(\mathbf{I} - \tfrac{\Delta t}{2}\Lop) \V^{j+1} = (\mathbf{I} + \tfrac{\Delta t}{2}\Lop) \V^j
\end{equation}

\subsubsection{Stability and Accuracy}

The Crank-Nicolson method offers:
\begin{itemize}
\item \textbf{Unconditional stability}: No time step restrictions
\item \textbf{Second-order accuracy}: $O((\Delta t)^2) + O((\Delta S)^2)$ truncation error
\item \textbf{Optimal convergence}: Best accuracy among finite difference methods
\end{itemize}

\subsubsection{Implementation Results}

\textbf{Computational Details:}
\begin{itemize}
\item Algorithm complexity: $O(MN)$ operations
\item Memory requirement: $O(M)$ for vectors plus two tridiagonal matrices
\item Computation time: $\sim\SI{0.041}{\second}$
\end{itemize}

\textbf{Numerical Results:}
\begin{itemize}
\item Option value: $V(S_0, 0) = \$4.5949$
\item Absolute error: $|4.5949 - 4.6150| = \$0.0201$
\item Relative error: $0.43\%$
\end{itemize}

\subsubsection{Algorithm Description}

The following presents the Crank-Nicolson finite difference method in structured algorithmic form.

\begin{algorithm}[H]
\caption{Crank-Nicolson Finite Difference for Black-Scholes}
\begin{algorithmic}[1]
\REQUIRE $S_0$ (initial stock price), $K$ (strike price), $T$ (time to expiration), $r$ (risk-free rate), $\sigma$ (volatility), $S_{max}$ (maximum stock price), $M$ (spatial grid points), $N$ (temporal grid points)
\ENSURE Option value $V(S_0, 0)$

\STATE $\Delta S \leftarrow S_{max} / M$
\STATE $\Delta t \leftarrow T / N$
\STATE $V \leftarrow$ \textsc{Initialize-Grid-CN}$(M, K, S_{max})$
\STATE $A \leftarrow$ \textsc{Build-Left-Matrix}$(M, \Delta t, \Delta S, \sigma, r, S_{max})$
\STATE $B \leftarrow$ \textsc{Build-Right-Matrix}$(M, \Delta t, \Delta S, \sigma, r, S_{max})$
\STATE \textsc{Set-Boundary-Conditions-CN}$(V, M, K, r, T, S_{max})$

\FOR{$j = N-1$ \TO $0$}
    \STATE $t \leftarrow j \cdot \Delta t$
    \STATE $b \leftarrow B \cdot V$ \COMMENT{Compute right-hand side}
    \STATE $V \leftarrow$ \textsc{Solve-Tridiagonal-System}$(A, b)$
    \STATE \textsc{Update-Boundary-Conditions-CN}$(V, M, K, r, t, S_{max})$
\ENDFOR

\RETURN \textsc{Interpolate-Value-CN}$(V, S_0, \Delta S)$
\end{algorithmic}
\end{algorithm}

\begin{algorithm}[H]
\caption{Build-Left-Matrix}
\begin{algorithmic}[1]
\REQUIRE $M$ (spatial points), $\Delta t$, $\Delta S$, $\sigma$, $r$, $S_{max}$
\ENSURE Left-hand side tridiagonal matrix $A$ of size $(M-1) \times (M-1)$

\STATE $A \leftarrow$ new tridiagonal matrix of size $(M-1) \times (M-1)$

\FOR{$i = 1$ \TO $M-1$}
    \STATE $S_i \leftarrow i \cdot \Delta S$
    \STATE $\alpha \leftarrow \frac{\Delta t}{4} \left( \frac{\sigma^2 S_i^2}{(\Delta S)^2} - \frac{rS_i}{\Delta S} \right)$
    \STATE $\beta \leftarrow 1 + \frac{\Delta t}{2} \left( \frac{\sigma^2 S_i^2}{(\Delta S)^2} + r \right)$
    \STATE $\gamma \leftarrow \frac{\Delta t}{4} \left( \frac{\sigma^2 S_i^2}{(\Delta S)^2} + \frac{rS_i}{\Delta S} \right)$
    
    \IF{$i > 1$}
        \STATE $A[i][i-1] \leftarrow -\alpha$ \COMMENT{Lower diagonal}
    \ENDIF
    \STATE $A[i][i] \leftarrow \beta$ \COMMENT{Main diagonal}
    \IF{$i < M-1$}
        \STATE $A[i][i+1] \leftarrow -\gamma$ \COMMENT{Upper diagonal}
    \ENDIF
\ENDFOR

\RETURN $A$
\end{algorithmic}
\end{algorithm}

\begin{algorithm}[H]
\caption{Build-Right-Matrix}
\begin{algorithmic}[1]
\REQUIRE $M$ (spatial points), $\Delta t$, $\Delta S$, $\sigma$, $r$, $S_{max}$
\ENSURE Right-hand side tridiagonal matrix $B$ of size $(M-1) \times (M-1)$

\STATE $B \leftarrow$ new tridiagonal matrix of size $(M-1) \times (M-1)$

\FOR{$i = 1$ \TO $M-1$}
    \STATE $S_i \leftarrow i \cdot \Delta S$
    \STATE $\alpha \leftarrow \frac{\Delta t}{4} \left( \frac{\sigma^2 S_i^2}{(\Delta S)^2} - \frac{rS_i}{\Delta S} \right)$
    \STATE $\beta \leftarrow 1 - \frac{\Delta t}{2} \left( \frac{\sigma^2 S_i^2}{(\Delta S)^2} + r \right)$
    \STATE $\gamma \leftarrow \frac{\Delta t}{4} \left( \frac{\sigma^2 S_i^2}{(\Delta S)^2} + \frac{rS_i}{\Delta S} \right)$
    
    \IF{$i > 1$}
        \STATE $B[i][i-1] \leftarrow \alpha$ \COMMENT{Lower diagonal}
    \ENDIF
    \STATE $B[i][i] \leftarrow \beta$ \COMMENT{Main diagonal}
    \IF{$i < M-1$}
        \STATE $B[i][i+1] \leftarrow \gamma$ \COMMENT{Upper diagonal}
    \ENDIF
\ENDFOR

\RETURN $B$
\end{algorithmic}
\end{algorithm}

\begin{algorithm}[H]
\caption{Initialize-Grid-CN}
\begin{algorithmic}[1]
\REQUIRE $M$ (spatial grid points), $K$ (strike price), $S_{max}$ (maximum stock price)
\ENSURE Initial grid $V$ with terminal conditions

\STATE $V \leftarrow$ new array of size $M+1$
\STATE $\Delta S \leftarrow S_{max} / M$

\FOR{$i = 0$ \TO $M$}
    \STATE $S_i \leftarrow i \cdot \Delta S$
    \STATE $V[i] \leftarrow \max(S_i - K, 0)$ \COMMENT{Call option payoff}
\ENDFOR

\RETURN $V$
\end{algorithmic}
\end{algorithm}

\begin{algorithm}[H]
\caption{Set-Boundary-Conditions-CN}
\begin{algorithmic}[1]
\REQUIRE Grid $V$, $M$ (spatial points), $K$ (strike), $r$ (rate), $T$ (time), $S_{max}$
\ENSURE Grid $V$ with boundary conditions set

\STATE $V[0] \leftarrow 0$ \COMMENT{Call worthless when $S=0$}
\STATE $V[M] \leftarrow S_{max} - K \cdot e^{-rT}$ \COMMENT{Call intrinsic at $S_{max}$}
\end{algorithmic}
\end{algorithm}

\begin{algorithm}[H]
\caption{Update-Boundary-Conditions-CN}
\begin{algorithmic}[1]
\REQUIRE Grid $V$, $M$ (spatial points), $K$ (strike), $r$ (rate), $t$ (current time), $S_{max}$
\ENSURE Grid $V$ with updated boundary conditions

\STATE $V[0] \leftarrow 0$ \COMMENT{Call worthless at $S=0$}
\STATE $V[M] \leftarrow S_{max} - K \cdot e^{-r\tau}$ \COMMENT{Call intrinsic at $S_{max}$}
\end{algorithmic}
\end{algorithm}

\paragraph{Complexity Analysis}

\textbf{Time Complexity:} The main algorithm executes $N$ time steps, each requiring $O(M)$ operations for matrix-vector multiplication and $O(M)$ operations for tridiagonal system solution, yielding $O(N \cdot M)$ total complexity.

\textbf{Space Complexity:} Requires $O(M)$ storage for the solution vector and $O(M)$ for each tridiagonal matrix (left and right matrices).

\section{Comprehensive Results and Analysis}

\subsection{Method Comparison Summary}

Table \ref{tab:comparison} presents a comprehensive comparison of all implemented methods using the standard test parameters.

\begin{table}[H]
\centering
\caption{Comprehensive Methods Comparison}
\label{tab:comparison}
\begin{tabular}{l S[table-format=1.4] S[table-format=1.4] S[table-format=1.2] S[table-format=1.3]}
 \toprule
 \textbf{Method} & {\textbf{Value (\$)}} & {\textbf{Abs.~Error (\$)}} & {\textbf{Rel.~Error (\%)}} & {\textbf{Time (s)}} \\
 \midrule
Analytical & \SI{4.6150}{\$} & \multicolumn{1}{c}{---} & \multicolumn{1}{c}{---} & \SI{0.001}{\second} \\
Explicit FD & \SI{4.5955}{\$} & \SI{0.0195}{\$} & \SI{0.42}{\percent} & \SI{0.124}{\second} \\
Implicit FD & \SI{4.5944}{\$} & \SI{0.0206}{\$} & \SI{0.45}{\percent} & \SI{0.052}{\second} \\
Crank-Nicolson & \SI{4.5949}{\$} & \SI{0.0201}{\$} & \SI{0.43}{\percent} & \SI{0.041}{\second} \\
 \bottomrule
\end{tabular}
 \end{table}

\subsection{Key Insights}

\subsubsection{Accuracy Assessment}

All numerical methods achieve excellent accuracy:
\begin{itemize}
\item \textbf{All methods exceed \SI{99.5}{\percent} accuracy} relative to the analytical solution
\item \textbf{Maximum absolute error}: \SI{0.0206}{\dollar} (implicit method)
\item \textbf{Maximum relative error}: \SI{0.45}{\percent} (implicit method)
\item \textbf{Best accuracy}: Explicit method with \SI{0.42}{\percent} relative error
\end{itemize}

\subsubsection{Performance Analysis}

\textbf{Computational Speed Ranking:}
\begin{enumerate}
\item Analytical: Fastest (baseline)
\item Crank-Nicolson: 41× slower than analytical
\item Implicit: 52× slower than analytical  
\item Explicit: 124× slower than analytical
\end{enumerate}

\textbf{Speed vs. Accuracy Trade-offs:}
\begin{itemize}
\item \textbf{Crank-Nicolson}: Best balance of speed and accuracy
\item \textbf{Implicit}: Good stability with moderate speed
\item \textbf{Explicit}: Simplest implementation but slowest
\end{itemize}

\subsection{Method Characteristics}

\subsubsection{Analytical Method}
\textbf{Advantages:}
\begin{itemize}
\item Exact solution (machine precision)
\item Fastest computation
\item No discretization errors
\end{itemize}

\textbf{Limitations:}
\begin{itemize}
\item Limited to standard European options
\item Cannot handle complex boundary conditions
\item Not applicable to path-dependent options
\end{itemize}

\subsubsection{Explicit Finite Difference}
\textbf{Advantages:}
\begin{itemize}
\item Simple implementation
\item Direct time stepping
\item Easy to understand and debug
\end{itemize}

\textbf{Disadvantages:}
\begin{itemize}
\item Stability constraints on time step
\item Requires small time steps
\item Slowest among numerical methods
\end{itemize}

\subsubsection{Implicit Finite Difference}
\textbf{Advantages:}
\begin{itemize}
\item Unconditionally stable
\item Can use larger time steps
\item Efficient tridiagonal solver
\end{itemize}

\textbf{Disadvantages:}
\begin{itemize}
\item Requires linear system solution
\item First-order accuracy in time
\item More complex implementation
\end{itemize}

\subsubsection{Crank-Nicolson Method}
\textbf{Advantages:}
\begin{itemize}
\item Second-order accurate in time and space
\item Unconditionally stable
\item Best balance of accuracy and stability
\item Industry standard for parabolic PDEs
\end{itemize}

\textbf{Disadvantages:}
\begin{itemize}
\item Most complex implementation
\item Requires two matrix operations per time step
\item Higher memory requirements
\end{itemize}

\section{Advanced Topics and Best Practices}

\subsection{Grid Refinement Analysis}

The accuracy of finite difference methods can be significantly improved through grid refinement. Our analysis demonstrates the effect of different grid resolutions:

\textbf{Coarse Grid (\M=\num{20}, \N=\num{25}):}
\begin{itemize}
\item Explicit method result: \SI{4.02}{\$}
\item Relative error: \SI{12.8}{\percent} (high due to coarse discretization)
\end{itemize}

\textbf{Refined Grid (\M=\num{100}, \N=\num{1000}):}
\begin{itemize}
\item Explicit method result: \SI{4.5955}{\$}
\item Relative error: \SI{0.42}{\percent} (excellent accuracy)
\end{itemize}

This demonstrates that \textbf{grid refinement dramatically improves accuracy}, reducing the error from \SI{12.8}{\percent} to \SI{0.42}{\percent}.

\subsection{Convergence Analysis}

\textbf{Theoretical Convergence Rates:}
\begin{itemize}
\item \textbf{Explicit}: $O(\Delta t) + O((\Delta S)^2)$
\item \textbf{Implicit}: $O(\Delta t) + O((\Delta S)^2)$
\item \textbf{Crank-Nicolson}: $O((\Delta t)^2) + O((\Delta S)^2)$
\end{itemize}

The Crank-Nicolson method's second-order time accuracy makes it the optimal choice for production applications.

\subsection{Practical Implementation Guidelines}

\subsubsection{Method Selection Criteria}

\textbf{Use Analytical Method when:}
\begin{itemize}
\item Pricing standard European options
\item Requiring exact results
\item Speed is critical
\end{itemize}

\textbf{Use Explicit Method when:}
\begin{itemize}
\item Learning finite difference concepts
\item Simple implementation is required
\item Debugging numerical schemes
\end{itemize}

\textbf{Use Implicit Method when:}
\begin{itemize}
\item Stability is more important than accuracy
\item Large time steps are needed
\item Memory is limited
\end{itemize}

\textbf{Use Crank-Nicolson when:}
\begin{itemize}
\item Maximum accuracy is required
\item Production applications
\item Complex option features need to be handled
\end{itemize}

\subsubsection{Grid Design Recommendations}

\textbf{Spatial Grid ($S$ direction):}
\begin{itemize}
\item Use at least 50-100 points across the domain
\item Concentrate points near the strike price
\item Ensure $S_{max} \geq 2-3 \times S_0$
\end{itemize}

\textbf{Temporal Grid ($t$ direction):}
\begin{itemize}
\item For explicit methods: ensure stability condition is satisfied
\item For implicit/Crank-Nicolson: choose based on accuracy requirements
\item Use at least 100-1000 time steps for smooth convergence
\end{itemize}

\subsection{Extensions and Applications}

\subsubsection{American Options}

Finite difference methods can be extended to American options by incorporating early exercise conditions:

\begin{equation}
V(S,t) = \max(S - K, V_{European}(S,t))
\end{equation}

This requires iterative solution techniques at each time step.

\subsubsection{Barrier Options}

Barrier options with knock-in or knock-out features can be handled by modifying boundary conditions:

\begin{itemize}
\item \textbf{Knock-out barriers}: Set $V = 0$ when barrier is hit
\item \textbf{Knock-in barriers}: Track rebate payments and activation conditions
\end{itemize}

\subsubsection{Multi-Asset Options}

The finite difference framework extends to multi-dimensional problems:

\begin{equation}
\frac{\partial V}{\partial t} + \sum_{i=1}^n \left( \frac{1}{2}\sigma_i^2 S_i^2 \frac{\partial^2 V}{\partial S_i^2} + r S_i \frac{\partial V}{\partial S_i} \right) + \sum_{i<j} \rho_{ij}\sigma_i\sigma_j S_i S_j \frac{\partial^2 V}{\partial S_i \partial S_j} - rV = 0
\end{equation}

\section{Computational Implementation}

\subsection{Software Architecture}

Our implementation consists of modular Python scripts:

\begin{itemize}
\item \texttt{black\_scholes\_analytical.py}: Exact Black-Scholes formula
\item \texttt{black\_scholes\_explicit.py}: Explicit finite difference implementation
\item \texttt{black\_scholes\_implicit.py}: Implicit finite difference with tridiagonal solver
\item \texttt{black\_scholes\_crank\_nicolson.py}: Crank-Nicolson implementation
\item \texttt{black\_scholes\_comparison.py}: Comprehensive comparison framework
\end{itemize}

\subsection{Key Implementation Features}

\subsubsection{Numerical Precision}

All computations use double-precision arithmetic with careful attention to:
\begin{itemize}
\item Cumulative normal distribution accuracy
\item Matrix conditioning for implicit methods
\item Interpolation accuracy for result extraction
\end{itemize}

\subsubsection{Boundary Condition Implementation}

\textbf{Lower Boundary ($S = 0$):}
\begin{equation}
V(0,t) = 0 \quad \text{(call option worthless)}
\end{equation}

\textbf{Upper Boundary ($S = S_{max}$):}
\begin{equation}
V(S_{max},t) = S_{max} - K e^{-r(T-t)} \quad \text{(linear extrapolation)}
\end{equation}

\subsubsection{Efficient Linear Algebra}

For implicit and Crank-Nicolson methods, we use:
\begin{itemize}
\item Sparse matrix representations for memory efficiency
\item Specialized tridiagonal solvers for $O(M)$ complexity
\item scipy.sparse.linalg routines for robust numerical solution
\end{itemize}

\section{Conclusion}

This comprehensive study of the Black-Scholes equation and its numerical solution methods provides several key insights:

\subsection{Main Findings}

\begin{enumerate}
\item \textbf{All numerical methods achieve excellent accuracy}: With proper grid refinement, explicit, implicit, and Crank-Nicolson methods all exceed \SI{99.5}{\percent} accuracy relative to the analytical solution.

\item \textbf{Crank-Nicolson is optimal for production use}: It offers the best balance of accuracy, stability, and computational efficiency, making it the preferred choice for practical applications.

\item \textbf{Grid refinement is crucial}: The difference between coarse and refined grids is dramatic, improving accuracy from \SI{12.8}{\percent} to \SI{0.42}{\percent} error.

\item \textbf{Method selection depends on requirements}: Each method has distinct advantages depending on the specific application needs.
\end{enumerate}

\subsection{Practical Recommendations}

For practitioners implementing Black-Scholes numerical methods:

\begin{itemize}
\item \textbf{Start with analytical solutions} when applicable for validation
\item \textbf{Use Crank-Nicolson for production systems} requiring maximum accuracy
\item \textbf{Implement proper grid refinement} to achieve desired accuracy levels
\item \textbf{Validate results against analytical benchmarks} when possible
\item \textbf{Consider computational requirements} when selecting methods
\end{itemize}

\subsection{Future Directions}

This foundation enables extension to more complex financial instruments:

\begin{itemize}
\item Path-dependent options (Asian, lookback)
\item Multi-asset derivatives (basket options, spread options)
\item Interest rate derivatives (caps, floors, swaptions)
\item Credit derivatives with jump-diffusion processes
\end{itemize}

The finite difference framework presented here provides a robust foundation for these advanced applications, demonstrating the power and flexibility of numerical methods in computational finance.

\subsection{Educational Impact}

This comprehensive study demonstrates the progression from theoretical understanding to practical implementation:

\begin{enumerate}
\item \textbf{Mathematical rigor}: Proper classification and analysis of the PDE
\item \textbf{Analytical benchmarks}: Exact solutions for validation
\item \textbf{Numerical approximation}: Multiple finite difference schemes
\item \textbf{Comparative analysis}: Objective evaluation of methods
\item \textbf{Practical implementation}: Real-world computational considerations
\end{enumerate}

This approach provides students and practitioners with both theoretical understanding and practical skills necessary for modern computational finance applications.

\section*{Acknowledgments}

This work synthesizes results from five intensive weeks of study covering partial differential equations in finance, Black-Scholes mathematical analysis, and finite difference implementations. The comprehensive computational results demonstrate the effectiveness of numerical methods for option pricing when properly implemented and validated.

\section*{References}

\begin{enumerate}
\item Black, F., \& Scholes, M. (1973). The pricing of options and corporate liabilities. \textit{Journal of Political Economy}, 81(3), 637-654.
\item Merton, R. C. (1973). Theory of rational option pricing. \textit{The Bell Journal of Economics and Management Science}, 4(1), 141-183.
\item Hull, J. C. (2017). \textit{Options, futures, and other derivatives}. Pearson.
\item Wilmott, P., Howison, S., \& Dewynne, J. (1995). \textit{The mathematics of financial derivatives: a student introduction}. Cambridge University Press.
\item Duffy, D. J. (2013). \textit{Finite difference methods in financial engineering: a partial differential equation approach}. John Wiley \& Sons.
\end{enumerate}

\end{document}
