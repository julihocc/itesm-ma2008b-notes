\documentclass[12pt,a4paper]{article}
\usepackage[utf8]{inputenc}
\usepackage[spanish,mexico]{babel}
\usepackage{amsmath}
\usepackage{amsfonts}
\usepackage{amssymb}
\usepackage[margin=2.5cm]{geometry}
\usepackage{enumerate}
\usepackage{icomma} % Para usar punto decimal en lugar de coma

\title{\textbf{Examen de Análisis Numérico}}
\author{}
\date{}

\begin{document}

\maketitle

\section*{Instrucciones}

\subsection*{Modalidad de Examen}
\begin{itemize}
\item \textbf{Resolución:} Manual, escrito a tinta en papel
\item \textbf{Formato:} Utilizar únicamente un lado de cada hoja (el reverso se reserva para anotaciones del profesor)
\item \textbf{Organización:} Cada ejercicio debe comenzar en una hoja nueva (no mezclar ejercicios en la misma hoja)
\item \textbf{Entrega:} Física en clase + archivo digital escaneado subido a Canvas
\end{itemize}

\subsection*{Evaluación}
\begin{itemize}
\item \textbf{Puntuación:} 10 puntos por inciso (100 puntos total)
\item \textbf{Criterios:} Las respuestas deben estar justificadas matemáticamente y ser numéricamente correctas
\end{itemize}

\subsection*{Herramientas Permitidas}
\begin{itemize}
\item \textbf{Calculadora:} Solo calculadoras científicas básicas (sin capacidad de programación, gráficas o simbólicas)
\item \textbf{Operaciones permitidas:} Aritmética básica (+, -, \(\times\), \(\div\)), funciones elementales (\(\ln\), \(\exp\), \(\sin\), \(\cos\), etc.)
\item \textbf{Prohibido:} Software de cálculo numérico, aplicaciones móviles, calculadoras programables
\end{itemize}

\subsection*{Formato de Presentación}
\begin{itemize}
\item \textbf{Cálculos intermedios y resultados finales:} Redondear a 3 cifras decimales
\item \textbf{Procedimiento:} Mostrar todos los pasos de cálculo de forma ordenada y legible
\item \textbf{Conclusión:} Al final de cada ejercicio, presentar los resultados numéricos en forma tabular como resumen
\end{itemize}

\section*{Ejercicio 1: Búsqueda de Raíces}

Considera la ecuación \(f(x) = x^4 - 6x^2 - 7 = 0\).

\subsection*{a) Solución exacta}
Encuentra las raíces reales de la ecuación utilizando un método algebraico. \textit{Sugerencia: Realiza la sustitución \(u = x^2\) para convertir la ecuación en una cuadrática.}

\subsection*{b) Método de Newton-Raphson}
Aplica el método de Newton-Raphson con valor inicial \(x_0 = 2.5\) para calcular tres iteraciones. Muestra todos los cálculos intermedios, incluyendo la obtención de la derivada \(f'(x)\) y la fórmula:

\[x_{n+1} = x_n - \frac{f(x_n)}{f'(x_n)}\]

\subsection*{c) Método de la Secante}
Aplica el método de la secante con valores iniciales \(x_0 = 2\) y \(x_1 = 3\) para calcular tres iteraciones. Muestra todos los cálculos usando la fórmula:

\[x_{n+1} = x_n - f(x_n) \cdot \frac{x_n - x_{n-1}}{f(x_n) - f(x_{n-1})}\]

\section*{Ejercicio 2: Integración Numérica}

Se desea aproximar la integral:

\[I = \int_{1}^{2} \frac{\ln(x)}{x} \, dx\]

\subsection*{a) Solución analítica}
Encuentra el valor exacto de la integral \(I\) definida utilizando el teorema fundamental del cálculo.

\subsection*{b) Regla del Trapecio}
Aplica la regla del trapecio compuesto con \(n = 4\) subintervalos para aproximar el valor de \(I\), y compara con el valor exacto.

\subsection*{c) Regla de Simpson}
Aplica la regla de Simpson compuesto con \(n = 4\) subintervalos para aproximar el valor de \(I\), y compara con el valor exacto.

\section*{Ejercicio 3: Ecuaciones Diferenciales Ordinarias}

Considera el problema de valor inicial:

\[\begin{cases}
y'(t) = 2y(t) - t, \quad t \in [0, 1] \\
y(0) = 1
\end{cases}\]

\subsection*{a) Solución Analítica}
Calcula la solución analítica exacta de esta ecuación diferencial y evalúa \(y(1)\).

\subsection*{b) Método de Euler}
Aplica el método de Euler con paso \(h = 0.25\) para aproximar \(y(1)\). Muestra todos los cálculos paso a paso usando la fórmula:

\[y_{n+1} = y_n + h \cdot f(t_n, y_n)\]

\subsection*{c) Método de Runge-Kutta de Orden 2}
Aplica el método de Runge-Kutta de segundo orden (método del punto medio) con paso \(h = 0.5\) para aproximar \(y(1)\). Muestra todos los cálculos detalladamente usando las fórmulas:

\begin{gather*}
k_1 = h \cdot f(t_n, y_n) \\
k_2 = h \cdot f\left(t_n + \frac{h}{2}, y_n + \frac{k_1}{2}\right) \\
y_{n+1} = y_n + k_2
\end{gather*}

\subsection*{d) Método de Runge-Kutta de Orden 4}
Aplica una iteración del método de Runge-Kutta de cuarto orden con paso \(h = 1\) para aproximar \(y(1)\) directamente. Muestra todos los cálculos usando las fórmulas:

\begin{gather*}
k_1 = h \cdot f(t_n, y_n) \\
k_2 = h \cdot f\left(t_n + \frac{h}{2}, y_n + \frac{k_1}{2}\right) \\
k_3 = h \cdot f\left(t_n + \frac{h}{2}, y_n + \frac{k_2}{2}\right) \\
k_4 = h \cdot f(t_n + h, y_n + k_3) \\
y_{n+1} = y_n + \frac{k_1 + 2k_2 + 2k_3 + k_4}{6}
\end{gather*}

\end{document}