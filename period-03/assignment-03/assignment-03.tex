\documentclass[12pt,a4paper]{article}
\usepackage[utf8]{inputenc}
\usepackage[spanish]{babel}
\usepackage{amsmath}
\usepackage{amsfonts}
\usepackage{amssymb}
\usepackage[margin=2.5cm]{geometry}
\usepackage{enumerate}

\title{\textbf{Examen de Análisis Numérico}}
\author{}
\date{}

\begin{document}

\maketitle

\section*{Instrucciones}

\subsection*{Modalidad de Examen}
\begin{itemize}
\item \textbf{Resolución:} Manual, escrito a tinta en papel
\item \textbf{Formato:} Utilizar únicamente un lado de cada hoja (el reverso se reserva para anotaciones del profesor)
\item \textbf{Organización:} Cada ejercicio debe comenzar en una hoja nueva (no mezclar ejercicios en la misma hoja)
\item \textbf{Entrega:} Física en clase + archivo digital escaneado subido a Canvas
\end{itemize}

\subsection*{Evaluación}
\begin{itemize}
\item \textbf{Puntuación:} 10 puntos por inciso (100 puntos total)
\item \textbf{Criterios:} Las respuestas deben estar justificadas matemáticamente y ser numéricamente correctas
\end{itemize}

\subsection*{Herramientas Permitidas}
\begin{itemize}
\item \textbf{Calculadora:} Solo calculadoras científicas básicas (sin capacidad de programación, gráficas o simbólicas)
\item \textbf{Operaciones permitidas:} Aritmética básica (+, -, $\times$, $\div$), funciones elementales ($\ln$, $\exp$, $\sin$, $\cos$, etc.)
\item \textbf{Prohibido:} Software de cálculo numérico, aplicaciones móviles, calculadoras programables
\end{itemize}

\subsection*{Formato de Presentación}
\begin{itemize}
\item \textbf{Cálculos intermedios:} Mantener mínimo 4 cifras decimales
\item \textbf{Resultados finales:} Redondear a 3 cifras decimales
\item \textbf{Procedimiento:} Mostrar todos los pasos de cálculo de forma ordenada y legible
\item \textbf{Conclusión:} Al final de cada ejercicio, presentar los resultados numéricos en forma tabular como resumen
\end{itemize}

\subsection*{Ejemplos de Tablas de Resultados}

\textbf{Ejercicio 1:}
\begin{center}
\begin{tabular}{|l|c|}
\hline
\textbf{Método} & \textbf{Resultado} \\
\hline
Newton-Raphson (2 iter.) & $x \approx$ \\
\hline
Secante (1 iter.) & $x \approx$ \\
\hline
\end{tabular}
\end{center}

\textbf{Ejercicio 2:}
\begin{center}
\begin{tabular}{|l|c|c|}
\hline
\textbf{Método} & \textbf{Aproximación} & \textbf{Error Absoluto} \\
\hline
Valor Exacto & & - \\
\hline
Trapecio ($n=4$) & & \\
\hline
Simpson ($n=4$) & & \\
\hline
\end{tabular}
\end{center}

\textbf{Ejercicio 3:}
\begin{center}
\begin{tabular}{|l|c|c|c|}
\hline
\textbf{Método} & \textbf{h} & \textbf{Punto Evaluado} & \textbf{Aproximación} \\
\hline
Analítica & - & $y(1)$ & \\
\hline
Euler & $0.25$ & $y(1)$ & \\
\hline
RK2 & $0.5$ & $y(1)$ & \\
\hline
RK4 & $1.0$ & $y(1)$ & \\
\hline
\end{tabular}
\end{center}

\vspace{0.5cm}
\hrule
\vspace{0.5cm}

\section*{Ejercicio 1: Búsqueda de Raíces}

Considera la ecuación $f(x) = x^3 - 2x - 5 = 0$.

\subsection*{a) Existencia de raíces}
Demuestra que existe una raíz en el intervalo $[2, 3]$.

\subsection*{b) Método de Newton-Raphson}
Aplica el método de Newton-Raphson con valor inicial $x_0 = 2.5$ para calcular dos iteraciones. Muestra todos los cálculos intermedios, incluyendo la obtención de la derivada $f'(x)$ y la fórmula:

$$x_{n+1} = x_n - \frac{f(x_n)}{f'(x_n)}$$

\subsection*{c) Método de la Secante}
Aplica el método de la secante con valores iniciales $x_0 = 2$ y $x_1 = 3$ para calcular una iteración. Muestra todos los cálculos usando la fórmula:

$$x_{n+1} = x_n - f(x_n) \cdot \frac{x_n - x_{n-1}}{f(x_n) - f(x_{n-1})}$$

\vspace{0.5cm}
\hrule
\vspace{0.5cm}

\section*{Ejercicio 2: Integración Numérica}

Se desea aproximar la integral:

$$I = \int_{1}^{2} \frac{\ln(x)}{x} \, dx$$

\subsection*{a) Solución analítica}
Encuentra el valor exacto de la integral $I$ definida utilizando el teorema fundamental del cálculo.

\subsection*{b) Regla del Trapecio}
Aplica la regla del trapecio compuesto con $n = 4$ subintervalos para aproximar el valor de $I$, y compara con el valor exacto.

\subsection*{c) Regla de Simpson}
Aplica la regla de Simpson compuesto con $n = 4$ subintervalos para aproximar el valor de $I$, y compara con el valor exacto.

\vspace{0.5cm}
\hrule
\vspace{0.5cm}

\section*{Ejercicio 3: Ecuaciones Diferenciales Ordinarias}

Considera el problema de valor inicial:

$$\begin{cases}
y'(t) = 2y(t) - t, \quad t \in [0, 1] \\
y(0) = 1
\end{cases}$$

\subsection*{a) Solución Analítica}
Calcula la solución analítica exacta de esta ecuación diferencial y evalúa $y(1)$. Compara numéricamente este valor con todas las aproximaciones obtenidas en los siguientes incisos (Método de Euler, RK2 y RK4).

\subsection*{b) Método de Euler}
Aplica el método de Euler con paso $h = 0.25$ para aproximar $y(1)$. Muestra todos los cálculos paso a paso usando la fórmula:

$$y_{n+1} = y_n + h \cdot f(t_n, y_n)$$

\subsection*{c) Método de Runge-Kutta de Orden 2}
Aplica el método de Runge-Kutta de segundo orden (método del punto medio) con paso $h = 0.5$ para aproximar $y(1)$. Muestra todos los cálculos detalladamente usando las fórmulas:

\begin{align}
k_1 &= h \cdot f(t_n, y_n) \\
k_2 &= h \cdot f\left(t_n + \frac{h}{2}, y_n + \frac{k_1}{2}\right) \\
y_{n+1} &= y_n + k_2
\end{align}

\subsection*{d) Método de Runge-Kutta de Orden 4}
Aplica una iteración del método de Runge-Kutta de cuarto orden con paso $h = 1$ para aproximar $y(1)$ directamente. Muestra todos los cálculos usando las fórmulas:

\begin{align}
k_1 &= h \cdot f(t_n, y_n) \\
k_2 &= h \cdot f\left(t_n + \frac{h}{2}, y_n + \frac{k_1}{2}\right) \\
k_3 &= h \cdot f\left(t_n + \frac{h}{2}, y_n + \frac{k_2}{2}\right) \\
k_4 &= h \cdot f(t_n + h, y_n + k_3) \\
y_{n+1} &= y_n + \frac{k_1 + 2k_2 + 2k_3 + k_4}{6}
\end{align}

\end{document}