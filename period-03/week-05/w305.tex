\documentclass[aspectratio=169]{beamer}
\usetheme{metropolis}
\usepackage{amsmath}
\usepackage{graphicx}
\usepackage{tikz}
\usepackage{booktabs}

\title{Finite Difference Methods for Black-Scholes - Part 3}
\subtitle{Advanced Topics and Best Practices}
\author{Lecturer}
\date{\today}

\begin{document}

\begin{frame}
\titlepage
\end{frame}

\begin{frame}
\frametitle{Outline - Part 3}
\tableofcontents
\end{frame}

\section{Advanced Grid Techniques}

\begin{frame}
\frametitle{Grid Refinement - The Problem}
\textbf{Issue with Fixed Grids}:
\begin{itemize}
\item Uniform spacing may be inefficient
\item Need more points near important regions
\item Want to concentrate accuracy where needed most
\end{itemize}

\textbf{Key Regions for Options}:
\begin{itemize}
\item Near the strike price \(S = K\)
\item Near current time \(t = 0\)
\item Near boundaries
\end{itemize}
\end{frame}

\begin{frame}
\frametitle{Non-Uniform Grids}
\textbf{Basic Idea}: Use different spacing in different regions

\textbf{Common Approaches}:
\begin{itemize}
\item \textbf{Geometric progression}: \(\Delta S_i = \Delta S_0 \times r^i\)
\item \textbf{Hyperbolic tangent}: Dense near center, sparse at edges
\item \textbf{Manual specification}: Define grid points explicitly
\end{itemize}

\textbf{Benefit}: Better accuracy with same number of points
\end{frame}

\begin{frame}
\frametitle{Logarithmic Transformation}
\textbf{Coordinate Change}: Use \(x = \ln(S)\) instead of \(S\)

\begin{block}{Transformed PDE}
\[\frac{\partial V}{\partial t} + \frac{1}{2}\sigma^2 \frac{\partial^2 V}{\partial x^2} + \left(r - \frac{1}{2}\sigma^2\right) \frac{\partial V}{\partial x} - rV = 0\]
\end{block}

\textbf{Advantages}:
\begin{itemize}
\item Constant coefficients (easier to solve)
\item Natural uniform grid in \(x\) space
\item Better numerical stability
\end{itemize}
\end{frame}

\section{Higher-Order Methods}

\begin{frame}
\frametitle{Higher-Order Methods - Motivation}
\textbf{Standard finite differences}:
\begin{itemize}
\item Second-order accuracy: \(O((\Delta S)^2)\)
\item Need many grid points for high accuracy
\item Can we do better?
\end{itemize}

\textbf{Higher-order methods}:
\begin{itemize}
\item Fourth-order accuracy: \(O((\Delta S)^4)\)
\item Exponential accuracy (spectral methods)
\item Better accuracy with fewer points
\end{itemize}
\end{frame}

\section{Best Practices and Conclusion}

\begin{frame}
\frametitle{Method Comparison Summary}
\begin{center}
\begin{tabular}{lccc}
\toprule
\textbf{Method} & \textbf{Speed} & \textbf{Flexibility} & \textbf{Accuracy} \\
\midrule
Finite Differences & Medium & High & High \\
Monte Carlo & Slow & Very High & Medium \\
Binomial Trees & Fast & Medium & Medium \\
Analytical Formulas & Very Fast & Low & Very High \\
\bottomrule
\end{tabular}
\end{center}

\textbf{When to Use Finite Differences}:
\begin{itemize}
\item Complex boundary conditions
\item American-style exercise
\item Path-dependent features
\end{itemize}
\end{frame}

\begin{frame}
\frametitle{Key Takeaways}
\textbf{Main Messages}:
\begin{enumerate}
\item \textbf{Finite differences} provide robust, flexible option pricing
\item \textbf{Crank-Nicolson} offers best balance of accuracy and stability  
\item \textbf{Grid design} is crucial for accurate results
\item \textbf{Central differences} are more accurate than forward/backward
\item \textbf{Exotic options} easily handled with appropriate modifications
\end{enumerate}
\end{frame}

\begin{frame}
\frametitle{Thank You}
\textbf{Questions and Discussion}

\vspace{1cm}

\textbf{Summary of Complete Series}:
\begin{itemize}
\item \textbf{Part 1}: Introduction and Explicit Method
\item \textbf{Part 2}: Implicit Methods and Comparison  
\item \textbf{Part 3}: Advanced Topics and Best Practices
\end{itemize}

\vspace{1cm}

\textbf{Contact Information}:
\begin{itemize}
\item Email: lecturer@university.edu
\item Office Hours: Wednesdays 2-4 PM
\end{itemize}
\end{frame}

\end{document}