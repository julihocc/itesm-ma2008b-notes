\documentclass{beamer}
\usetheme{default}
\title{Partial Differential Equations in Finance}
\author{Your Name}
\date{\today}

\begin{document}

\frame{\titlepage}

\section{Introduction to PDEs in Financial Applications}

\begin{frame}{Overview of Topics}
    \begin{itemize}
        \item Charting theory and nature of boundary and initial conditions
        \item Explicit solutions, including original Black-Scholes formula
        \item Special problems arising when there are free boundaries
    \end{itemize}
\end{frame}

\begin{frame}{Key Questions in Financial PDEs}
    \begin{itemize}
        \item Physical interpretation of the equations?
        \item Mathematical properties of the solution?
        \item Techniques for obtaining explicit solutions?
    \end{itemize}
    
    \vspace{1cm}
    PDEs in finance:
    \begin{itemize}
        \item Fundamental equations (e.g., Black-Scholes)
        \item Linear vs. nonlinear problems
    \end{itemize}
\end{frame}

\begin{frame}{Considerations for PDEs in Finance}
    \begin{enumerate}
        \item Does the equation make sense as a well-posed problem?
        \begin{itemize}
            \item Appropriate boundary or initial/final conditions?
            \item Nature of the mathematical problem?
            \item Smooth or discontinuous solutions?
        \end{itemize}
        
        \item Can we develop analytical tools to solve the equation?
        
        \item How should we solve the equation numerically if necessary?
    \end{enumerate}
\end{frame}

\section{First Order Linear PDEs}

\begin{frame}{First Order Linear PDE}
    Consider the equation:
    \[
    \alpha(s,t)\frac{\partial u}{\partial s} + \beta(s,t)\frac{\partial u}{\partial t} = \gamma(s,t)u(t)
    \]
    
    \begin{block}{Linearity Property}
        If $u_1, u_2$ are solutions, then $c_1u_1 + c_2u_2$ is also a solution for constants $c_1,c_2$.
    \end{block}
    
    \begin{proof}
        \small
        Substitute $u = c_1u_1 + c_2u_2$ into the PDE and use linearity of derivatives.
    \end{proof}
\end{frame}

\begin{frame}{Constant Coefficient Case}
    When $\alpha,\beta$ are constants and $\gamma \equiv 0$:
    \[
    \alpha_0\frac{\partial u}{\partial s} + \beta_0\frac{\partial u}{\partial t} = 0 \quad (4.2)
    \]
    
    Define the vector $\vec{v} = (\alpha_0, \beta_0)$ with $\beta_0\alpha_0 \neq 0$.
    
    The directional derivative:
    \[
    \nabla_{\vec{v}} u = \langle \alpha_0, \beta_0 \rangle \cdot \langle \frac{\partial u}{\partial s}, \frac{\partial u}{\partial t} \rangle
    \]
    
    Thus, (4.2) $\Leftrightarrow \nabla_{\vec{v}} u \equiv 0$
\end{frame}

\begin{frame}{Characteristic Coordinates}
    Introduce new coordinates:
    \[
    \xi = \beta_0s + \alpha_0t, \quad \zeta = \beta_0s - \alpha_0t
    \]
    
    The Jacobian determinant:
    \[
    \begin{vmatrix}
    \beta_0 & \alpha_0 \\
    \beta_0 & -\alpha_0
    \end{vmatrix} = -2\alpha_0\beta_0 \neq 0
    \]
    
    Computing $\frac{\partial u}{\partial \xi}$:
    \[
    \frac{\partial u}{\partial \xi} = \beta_0\frac{\partial u}{\partial s} + \alpha_0\frac{\partial u}{\partial t} \equiv 0
    \]
    
    Therefore, $u$ depends only on $\zeta$: $u = F(\zeta)$
\end{frame}

\begin{frame}{Characteristics and Information Propagation}
    \begin{itemize}
        \item $\xi = \beta_0s - \alpha_0t$ is called a \textbf{characteristic}
        \item Characteristics represent directions in which information propagates
        \item Analogous to constants of integration in ODEs
    \end{itemize}
    
    \begin{alertblock}{Important Note}
        Without boundary conditions, the solution remains arbitrary (like the constant $C$ in ODEs)
    \end{alertblock}
\end{frame}



\end{document}