\documentclass{beamer}
\usetheme{default}
\title{Partial Differential Equations in Finance}
\author{Your Name}
\date{\today}

\begin{document}

\frame{\titlepage}

\section{Introduction to PDEs in Financial Applications}

\begin{frame}{Overview of Topics}
    \begin{itemize}
        \item Charting theory and nature of boundary and initial conditions
        \item Explicit solutions, including original Black-Scholes formula
        \item Special problems arising when there are free boundaries
    \end{itemize}
\end{frame}

\begin{frame}{Key Questions in Financial PDEs}
    \begin{itemize}
        \item Physical interpretation of the equations?
        \item Mathematical properties of the solution?
        \item Techniques for obtaining explicit solutions?
    \end{itemize}
    
    \vspace{1cm}
    PDEs in finance:
    \begin{itemize}
        \item Fundamental equations (e.g., Black-Scholes)
        \item Linear vs. nonlinear problems
    \end{itemize}
\end{frame}

\begin{frame}{Considerations for PDEs in Finance}
    \begin{enumerate}
        \item Does the equation make sense as a well-posed problem?
        \begin{itemize}
            \item Appropriate boundary or initial/final conditions?
            \item Nature of the mathematical problem?
            \item Smooth or discontinuous solutions?
        \end{itemize}
        
        \item Can we develop analytical tools to solve the equation?
        
        \item How should we solve the equation numerically if necessary?
    \end{enumerate}
\end{frame}

\section{First Order Linear PDEs}

\begin{frame}{First Order Linear PDE}
    Consider the equation:
    \[
    \alpha(s,t)\frac{\partial u}{\partial s} + \beta(s,t)\frac{\partial u}{\partial t} = \gamma(s,t)u(t)
    \]
    
    \begin{block}{Linearity Property}
        If $u_1, u_2$ are solutions, then $c_1u_1 + c_2u_2$ is also a solution for constants $c_1,c_2$.
    \end{block}
    
    \begin{proof}
        \small
        Substitute $u = c_1u_1 + c_2u_2$ into the PDE and use linearity of derivatives.
    \end{proof}
\end{frame}

\begin{frame}{Constant Coefficient Case}
    When $\alpha,\beta$ are constants and $\gamma \equiv 0$:
    \[
    \alpha_0\frac{\partial u}{\partial s} + \beta_0\frac{\partial u}{\partial t} = 0 \quad (4.2)
    \]
    
    Define the vector $\vec{v} = (\alpha_0, \beta_0)$ with $\beta_0\alpha_0 \neq 0$.
    
    The directional derivative:
    \[
    \nabla_{\vec{v}} u = \langle \alpha_0, \beta_0 \rangle \cdot \langle \frac{\partial u}{\partial s}, \frac{\partial u}{\partial t} \rangle
    \]
    
    Thus, (4.2) $\Leftrightarrow \nabla_{\vec{v}} u \equiv 0$
\end{frame}

\begin{frame}{Characteristic Coordinates}
    Introduce new coordinates:
    \[
    \xi = \beta_0s + \alpha_0t, \quad \zeta = \beta_0s - \alpha_0t
    \]
    
    The Jacobian determinant:
    \[
    \begin{vmatrix}
    \beta_0 & \alpha_0 \\
    \beta_0 & -\alpha_0
    \end{vmatrix} = -2\alpha_0\beta_0 \neq 0
    \]
    
    Computing $\frac{\partial u}{\partial \xi}$:
    \[
    \frac{\partial u}{\partial \xi} = \beta_0\frac{\partial u}{\partial s} + \alpha_0\frac{\partial u}{\partial t} \equiv 0
    \]
    
    Therefore, $u$ depends only on $\zeta$: $u = F(\zeta)$
\end{frame}

\begin{frame}{Characteristics and Information Propagation}
    \begin{itemize}
        \item $\xi = \beta_0s - \alpha_0t$ is called a \textbf{characteristic}
        \item Characteristics represent directions in which information propagates
        \item Analogous to constants of integration in ODEs
    \end{itemize}
    
    \begin{alertblock}{Important Note}
        Without boundary conditions, the solution remains arbitrary (like the constant $C$ in ODEs)
    \end{alertblock}
\end{frame}


\section{Second-Order Linear PDEs}

\begin{frame}{General Second-Order Linear PDE}
    Consider the general form in two variables:
    \[
    a(x,t)u_{xx} + 2b(x,t)u_{xt} + c(x,t)u_{tt} + d(x,t)u_x + e(x,t)u_t + f(x,t)u = g(x,t)
    \]
    where $a,b,c,d,e,f,g: \mathbb{R} \times \mathbb{R} \rightarrow \mathbb{R}$ are coefficient functions.
    
    \begin{block}{Principal Part}
        The highest-order terms form the principal part:
        \[
        L_p(u) = a u_{xx} + 2b u_{xt} + c u_{tt}
        \]
    \end{block}
\end{frame}

\begin{frame}{Classification of PDEs}
    Define the discriminant:
    \[
    \Delta(L) = b^2 - ac : (x,t) \mapsto b(x,t)^2 - a(x,t)c(x,t)
    \]
    
    The PDE is classified as:
    \begin{itemize}
        \item \textbf{Hyperbolic} if $\Delta(L) > 0$ (e.g., wave equation)
        \item \textbf{Parabolic} if $\Delta(L) = 0$ (e.g., heat equation)
        \item \textbf{Elliptic} if $\Delta(L) < 0$ (e.g., Laplace equation)
    \end{itemize}
\end{frame}

\begin{frame}{Coordinate Transformations}
    \begin{definition}
        A transformation $(\xi, \eta) = (\xi(x,t), \eta(x,t))$ is called a change of coordinates if the Jacobian is non-zero:
        \[
        J := \xi_x \eta_t - \xi_t \eta_x \neq 0
        \]
    \end{definition}
    
    \begin{theorem}
        The type of a linear second-order PDE in two variables is invariant under a change of coordinates.
    \end{theorem}
\end{frame}

\begin{frame}{Proof of Invariance}
    Under transformation $(\xi, \eta)$, the principal part becomes:
    \[
    A u_{\xi\xi} + 2B u_{\xi\eta} + C u_{\eta\eta}
    \]
    where:
    \[
    \begin{aligned}
        A &= a \xi_x^2 + 2b \xi_x \xi_t + c \xi_t^2 \\
        B &= a \xi_x \eta_x + b (\xi_x \eta_t + \xi_t \eta_x) + c \xi_t \eta_t \\
        C &= a \eta_x^2 + 2b \eta_x \eta_t + c \eta_t^2
    \end{aligned}
    \]
    
    The new discriminant satisfies:
    \[
    \Delta_{new} = AC - B^2 = J^2 (b^2 - ac) = J^2 \Delta_{original}
    \]
\end{frame}

\begin{frame}{Canonical Forms}
    \begin{itemize}
        \item \textbf{Hyperbolic}: $u_{\xi\xi} - u_{\eta\eta} = \text{lower order terms}$ (wave equation)
        \item \textbf{Parabolic}: $u_{\xi\xi} = \text{lower order terms}$ (heat equation)
        \item \textbf{Elliptic}: $u_{\xi\xi} + u_{\eta\eta} = \text{lower order terms}$ (Laplace equation)
    \end{itemize}
    
    \begin{examples}
        \begin{itemize}
            \item Wave equation: $u_{tt} = c^2(u_{xx} + u_{yy})$
            \item Heat equation: $u_t = k(u_{xx} + u_{yy})$
            \item Laplace equation: $u_{xx} + u_{yy} = 0$
        \end{itemize}
    \end{examples}
\end{frame}

\begin{frame}{Simplification Through Coordinates}
    A proper linear change of coordinates can transform:
    \begin{itemize}
        \item Hyperbolic equations to $u_{\xi\eta} = 0$
        \item Parabolic equations to $u_{\xi\xi} = 0$
        \item Elliptic equations to $u_{\xi\xi} + u_{\eta\eta} = 0$
    \end{itemize}
    
    \begin{alertblock}{Important Note}
        The classification determines:
        \begin{itemize}
            \item Nature of characteristics
            \item Appropriate boundary conditions
            \item Solution techniques
        \end{itemize}
    \end{alertblock}
\end{frame}


\section{Heat Equation Analysis}

\begin{frame}{Heat Equation}
    The fundamental heat equation:
    \[
    \frac{\partial u}{\partial t} = \frac{\partial^2 u}{\partial x^2}
    \]
    
    Physical interpretation:
    \begin{itemize}
        \item $u(x,t)$ represents temperature distribution along a thin insulated rod
        \item Models heat diffusion over time
    \end{itemize}
\end{frame}

\begin{frame}{Mathematical Properties}
    Key characteristics:
    \begin{itemize}
        \item Linear equation (superposition principle applies)
        \item Second order in space ($x$)
        \item First order in time ($t$)
        \item Parabolic type (discriminant $\Delta = 0$)
    \end{itemize}
    
    \begin{block}{Analytic Solutions}
        Solutions are analytic functions of $x$:
        \begin{itemize}
            \item For any $t^* > 0$ and $a < x_0 < b$
            \item $u(x,t^*)$ can be expressed as a convergent power series in $(x-x_0)$
        \end{itemize}
    \end{block}
\end{frame}

\begin{frame}{Solution Behavior}
    Important features:
    \begin{itemize}
        \item \textbf{Smoothing property}: Solutions become instantly smooth for $t>0$
        \item \textbf{Infinite propagation speed}: Initial disturbances spread infinitely fast
        \item \textbf{Maximum principle}: Extreme values occur only at initial/boundary conditions
    \end{itemize}
    
    \begin{alertblock}{Financial Interpretation}
        Analogous to:
        \begin{itemize}
            \item Diffusion of volatility in option pricing
            \item Smoothing of price shocks over time
        \end{itemize}
    \end{alertblock}
\end{frame}


\section{Initial Value Problems with Boundary Conditions}

\begin{frame}{Dirichlet Boundary Value Problem}
    Heat equation on finite interval:
    \[
    \frac{\partial u}{\partial t} = \frac{\partial^2 u}{\partial x^2}, \quad -L < x < L, \ t > 0
    \]
    
    With conditions:
    \begin{itemize}
        \item Initial condition: $u(x,0) = u_0(x)$
        \item Boundary conditions:
        \begin{itemize}
            \item $u(-L,t) = g_1(t)$ (left end temperature)
            \item $u(L,t) = g_2(t)$ (right end temperature)
        \end{itemize}
    \end{itemize}
    
    \begin{block}{Physical Interpretation}
        Models heat diffusion in a finite rod with controlled end temperatures
    \end{block}
\end{frame}

\begin{frame}{Neumann Boundary Value Problem}
    Same heat equation with flux conditions:
    \[
    \frac{\partial u}{\partial t} = \frac{\partial^2 u}{\partial x^2}, \quad -L < x < L, \ t > 0
    \]
    
    With conditions:
    \begin{itemize}
        \item Initial condition: $u(x,0) = u_0(x)$
        \item Flux boundary conditions:
        \begin{itemize}
            \item $-\frac{\partial u}{\partial x}(-L,t) = h_1(t)$ (left end flux)
            \item $\frac{\partial u}{\partial x}(L,t) = h_2(t)$ (right end flux)
        \end{itemize}
    \end{itemize}
    
    \begin{block}{Physical Interpretation}
        Models heat diffusion with controlled heat flow at boundaries
    \end{block}
\end{frame}

\begin{frame}{Infinite Domain Problem}
    Taking $L \to \infty$:
    \[
    \frac{\partial u}{\partial t} = \frac{\partial^2 u}{\partial x^2}, \quad x \in \mathbb{R}, \ t > 0
    \]
    
    With initial condition:
    \[
    u(x,0) = u_0(x)
    \]
    
    \begin{block}{Growth Conditions}
        For well-posedness:
        \begin{itemize}
            \item $u_0(x)$ piecewise continuous (finite jumps allowed)
            \item $\lim_{|x|\to\infty} \frac{u_0(x)}{e^{ax^2}} = 0 \quad \forall a > 0$
            \item $\lim_{|x|\to\infty} \frac{u(x,t)}{e^{ax^2}} = 0 \quad \forall a,t > 0$
        \end{itemize}
    \end{block}
\end{frame}

\begin{frame}{Semi-Infinite Domain with Mixed Conditions}
    Example problem:
    \[
    u_t = u_{xx}, \quad 0 < x < \infty, \ t > 0
    \]
    
    With conditions:
    \begin{itemize}
        \item Initial condition: $u(x,0) = u_0(x)$ (sufficiently smooth)
        \item Boundary condition: $u(0,t) = g(t)$
        \item Growth conditions at infinity
    \end{itemize}
    
    \begin{alertblock}{Well-Posedness}
        The problem is well-posed when:
        \begin{itemize}
            \item Initial/boundary conditions are compatible
            \item Growth conditions are satisfied
            \item Solution exists, is unique, and depends continuously on data
        \end{itemize}
    \end{alertblock}
\end{frame}

\end{document}